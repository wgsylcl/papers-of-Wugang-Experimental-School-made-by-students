\documentclass[10pt]{article}
\usepackage{mathpaper}
\usepackage{tabularx}

\begin{document}
\showsecret
\papertitle{第21$\boldsymbol{\sim}$23章综合能力提升卷}
\centerline{\Large 参考答案及评分标准}
\textbf{\selectingintroduction}
\begin{table}[!htb]
    \centering
    \begin{tabularx}{\textwidth}{|*{11}{>{\centering\arraybackslash}X|}} \hline
        题号 & 1 & 2 & 3 & 4 & 5 & 6 & 7 & 8 & 9 & 10 \\ \hline
        选项 & C & A & C & D & B & C & D & A & A & A \\ \hline
    \end{tabularx}
\end{table}

\par \textbf{\complitingintroduction}
\begin{table}[!htb]
    \centering
    \renewcommand\arraystretch{1.5}
    \begin{tabularx}{\textwidth}{*{3}{>{\centering\arraybackslash}X}}
        11.$(-1,-3)$ & 12.$(-3,-5)$ & 13.$-10$或$2$ \\
        14.$75^{\circ}$ & 15.\circnum{2} & 16.$5881$  \\
    \end{tabularx}
\end{table}

\setcounter{taskcounter}{16}

\begin{questions}{\answeringintroduction}
    \question \begin{subquestions} \subquestion
        $\because y=x^2-8x+12=(x-4)^2-4 \ \ \ \ \ \ \therefore \text{抛物线开口向上,且顶点为}(4,-4)$ \point{2} \par
        令$y=0$,$\because (x-4)^2-4=0 \Rightarrow (x-4)^2=4 \Rightarrow x-4=\pm 2,\therefore x_1=2$、$x_2=6$,即原抛物线与$x$轴的交点为$(2,0)$和$(6,0)$ \point{4}.
    \subquestion
    $\because y=-2x^2+6x+8=-2\left(x-\frac{3}{2}\right)^2+\frac{25}{2} \ \ \ \ \ \ \therefore \text{抛物线开口向下,且顶点为}\left(\frac{3}{2},\frac{25}{2}\right)$ \point{6} \par
    令$y=0$,$\because -2\left(x-\frac{3}{2}\right)^2+\frac{25}{2}=0 \Rightarrow \left(x-\frac{3}{2}\right)^2=\frac{25}{4} \Rightarrow x-\frac{3}{2}=\pm \frac{5}{2},\therefore x_1=4$、$x_2=-1$,即原抛物线与$x$轴的交点为$(4,0)$和$(-1,0)$ \point{8}.
    \end{subquestions}
    \question %18
    \begin{subquestions}
        \subquestion $$\because \Delta ABC、\Delta AEF均是等边三角形$$
        $$\therefore AB=AC、AE=AF、\angle BAC=\angle EAF=60^{\circ}$$
        $$\therefore \angle BAC + \angle CAE=\angle EAF + \angle CAE \Rightarrow \angle BAE = \angle CAF \point{2}$$
        $$\therefore \Delta BAE \text{\scalebox{-1}[1]{$\cong$}} \Delta CAF \ (SAS) \Rightarrow BE=CF \point{4}$$
        \subquestion $$\because \text{由(1),}\Delta BAE \text{\scalebox{-1}[1]{$\cong$}} \Delta CAF$$
        $$\therefore \angle ABE=\angle ACF$$
        $$\therefore \angle BAC=\angle BQC=60^{\circ} \point{6}$$
        $$\therefore \angle BQF=180^{\circ}-\angle BQC=120^{\circ} \point{8}$$
    \end{subquestions}
    \question %19
    \begin{subquestions}
        \subquestion 设平均在每轮传染中,每个流感患者可以传染流感给$x$人,依题,一次传染后共$(x+1)$人患上了流感,两次后共$(x+1)^2$人患上了流感,则
        $$\begin{aligned}
            (x+1)^2 &= 64 \\
            x+1 &= \pm 8 \\
            x_1 = 7 、x_2&=-9(负值舍去)
        \end{aligned}$$
        答:平均在每轮传染中,每个流感患者可以传染流感给$7$人 \point{4}.
        \subquestion 设平均在每轮传染中,每个流感患者可以传染流感给$y$人,依题,第一轮有$y$名患者,第二轮有$y^2$名患者,则
        $$\begin{aligned}
            y^2+y+1 &= 31 \\
            y^2+y-30 &= 0 \\
            (y+6)(y-5) &= 0 \\
            故y_1=5、y_2&=-6(负值舍去)
        \end{aligned}$$
        答:平均在每轮传染中,每个流感患者可以传染流感给$5$人 \point{8}.
    \end{subquestions}
    \question 由题,在关于$x$的一元二次方程$x^2+4x+m+1=0$中,$A=1$、$B=4$、$C=m+1$. \point{2}
    \begin{subquestions}
        \subquestion $\because$原方程有两个实数根 \par
        $\therefore \Delta \geq 0 \Rightarrow B^2-4AC \geq 0 \Rightarrow 16-4(m+1) \geq 0 \Rightarrow m \leq 3$ \point{4}
        \subquestion 由韦达定理,
        $\begin{cases}
            x_1+x_2=-\frac{B}{A}=-4 \\
            x_1x_2 = \frac{C}{A} = m+1
        \end{cases}$\point{6},
        依题,${x_1}^2+{x_2}^2-4x_1x_2=0$,则
        $$\begin{aligned}
            (x_1+x_2)^2-6x_1x_2 &= 0 \\
            16-6(m+1) &= 0 \\
            m &= \frac{5}{3} (满足 m \leq 3 \point{7})
        \end{aligned}$$
        综上所述,$m = \frac{5}{3}$. \point{8}
    \end{subquestions}
    \question \begin{subquestions}
        \subquestion 如图1,直线$l$即为所求. \point{2}
        \subquestion 如图2,点$E$即为所求. \point{5}
        \subquestion 如图2,点$F$即为所求. \point{8}
    \end{subquestions}
    \begin{figure}[!htb]
        \centering
        \subfigure[(1)]{
            \begin{tikzpicture}[scale=0.6]
                \draw[step=1,dashed] (-3,-3) grid (5,5);
                \coordinate[label=below right:{$C$}] (C) at (0,0);
                \coordinate[label=above left:{$A$}] (A) at (4,3);
                \coordinate[label=above right:{$B$}] (B) at (0,3);
                \draw (A) -- (B) -- (C) -- cycle;
                \draw[ansline] (A) -- (0,5) -- (B);
                \draw[ansline,blue] (C) -- (2.25,4.5) node[right]{$l$};
            \end{tikzpicture}}\qquad\qquad
        \subfigure[(2)]{
            \begin{tikzpicture}[scale=0.6]
                \draw[step=1,dashed] (-3,-3) grid (5,5);
                \coordinate[label=below right:{$C$}] (C) at (0,0);
                \coordinate[label=above left:{$A$}] (A) at (4,3);
                \coordinate[label=above right:{$B$}] (B) at (0,3);
                \coordinate[label=above left:{$T$}] (T) at (1.24848,0.93636);
                \filldraw (T) circle (0.064);
                \draw (A) -- (B) -- (C) -- cycle;
                \coordinate[label=above right:{$E$}] (E) at (0.5606,3);
                \draw[ansline,blue] (C) -- (E);
                \draw[ansline] (C) -- (-1,3) -- (B);
                \draw[ansline] (A) -- (-2,1);
                \draw[ansline] (-1,3) -- (T);
                \draw[ansline] (0,-3) -- (4,0);
                \draw[ansline] (B) -- (3.5609,-1.74787);
                \coordinate (Bt) at (2.88,-0.84);
                \draw[ansline] (C) -- (Bt) -- (A);
                \draw[ansline] (Bt) -- (2.33001,-2.7257);
                \draw[ansline] (-1,3) -- (2.9844,-2.31253);
                \draw[ansline] (E) -- (2.6,-1.8);
                \coordinate[label=right:{$F$}] (F) at (3.03697,-0.30182);
                \draw[ansline,blue] (-1,3) -- (F);
            \end{tikzpicture}
        }
    \end{figure}
    \question %22
    \begin{subquestions}
        \subquestion $x=100t \point{1}$、$h=5t^2 \point{2}$.
        \subquestion 由(1),$\because x=100t$,无人机飞行速度为$100$m/s \quad $\therefore$无人机和航弹与山脚间水平距离始终相等 \par
        $\therefore$航弹在山脚(即原点)处爆炸\point{4}又$\because$无人机在离地$500$m高度投弹$\therefore$下落距离为$500$m,即
        $$\begin{aligned}
            5t^2 &= 500 \\
            t^2 &= 100 \\
            t &= \pm 10 \\
            t_1 = 10、t_2&=-10(负值舍去)
        \end{aligned}$$
        $\therefore$航弹水平飞行距离$x=100t=1000$(m) \par
        答:投弹点水平距离山脚$1000$m. \point{6}
        \subquestion 将$x=100t$代入$h=5t^2$中可得$h=\frac{x^2}{2000}$,由于无人机在离地$500$m高度投弹,故设投弹点水平距离山脚$s$m,则投弹点坐标为$(-s,500)$,于是有
        $$y=-\frac{(x+s)^2}{2000}+500 \point{7}$$
        依题,由于山的坡度是$45^{\circ}$,攻击区域是距离山脚水平$100$m至$200$m的地方,故点$(100,100)$必须在航弹轨迹的下方(可在轨迹上)、点$(200,200)$必须在航弹轨迹的上方(可在轨迹上),即
        $$\begin{cases}
            y|_{x=100} = -\frac{(s+100)^2}{2000}+500 \geq 100 \ \ \ \ \ \text{\circnum{1}} \\
            y|_{x=200} = -\frac{(s+200)^2}{2000}+500 \leq 200 \ \ \ \ \ \text{\circnum{2}}
        \end{cases} \point{8} $$
        解不等式\circnum{1}
        $$\begin{aligned}
            -\frac{(s+100)^2}{2000}+500 & \geq 100 \\
            \frac{(s+100)^2}{2000} & \leq 400 \\
            (s+100)^2 & \leq 8 \times 10^5 \\
            -400\sqrt{5} &\leq s + 100 \leq 400\sqrt{5} \\
            -100-400\sqrt{5} &\leq s \leq 400\sqrt{5}-100
        \end{aligned}$$
        解不等式\circnum{2}
        $$\begin{aligned}
            -\frac{(s+200)^2}{2000}+500 & \leq 200 \\
            \frac{(s+200)^2}{2000} & \geq 300 \\
            (s+200)^2 & \geq 6 \times 10^5
        \end{aligned}$$
        $$\begin{aligned}
            s+200 \geq 200\sqrt{15} &或 s+200 \leq -200\sqrt{15} \\
            s \geq 200\sqrt{15}-200 &或 s \leq -200\sqrt{15}-200
        \end{aligned}$$
        于是原不等式组解集为
        $$200\sqrt{15}-200 \leq s \leq 400\sqrt{5}-100$$
        答:无人机与山脚间水平距离不低于$(200\sqrt{15}-200)$m,不多于$(400\sqrt{5}-100)$m.\point{10}
        \begin{figure}[!htb]
            \raggedleft
            \begin{tikzpicture}[>=Stealth,scale=0.5]
                \draw[->] (-7.5,0) -- (4.2,0) node[below] {$x$};
                \draw[->] (0,-1) -- (0,5.5) node[right] {$y$};
                \coordinate[label=below right:{$O$}] (O) at (0,0);
                \draw (0,0) -- (3.5,3.5) -- (4.2,3.5);
                \draw (-7,5) parabola bend (-7,5) (1.5,1.5);
                \filldraw (-7,5) circle (.1);
                \node[below] at (-7,5) {无人机};
                \draw[line width =2pt] (1,1) -- (2,2);
            \end{tikzpicture}
        \end{figure}
    \end{subquestions}
    \question %23
    \begin{subquestions}
        \subquestion $\because \Delta ABC$是等边三角形$\therefore AB=BC=CA$ \par
        如图,延长$CP$交$AB$于点$Q$,于是
        $$在\Delta AQC中,有AQ+AC>QC$$
        $$在\Delta BQP中,有BQ+QP>BP$$
        上面二式相加,得到
        $$\begin{aligned}
            AQ+AC+BQ+QP &> QC+BP \\
            AB+AC+QP &> BP+QP+PC \\
            AB+AC &> BP+PC \\
            BP+PC &< 2BC \point{2}
        \end{aligned}$$
        \begin{figure}[!htb]
            \raggedleft
            \begin{tikzpicture}[scale=0.45]
                \coordinate[label=above:{$A$}] (A) at (3.49,6.05);
                \coordinate[label=below left:{$B$}] (B) at (0,0);
                \coordinate[label=below right:{$C$}] (C) at (6.98,0);
                \draw (A) -- (B) -- (C) -- cycle;
                \coordinate[label=above:{$P$}] (P) at (3.99,3.01);
                \draw (B) -- (P) -- (C);
                \coordinate[label=left:{$Q$}] (Q) at (2.56566,4.44386);
                \draw[ansline] (P) -- (Q);
            \end{tikzpicture}
        \end{figure}\newpage
        \subquestion
        如图,连$AP$,将$\Delta APC$绕点$A$逆时针旋转至$\Delta ATB$,使$AC$与$AB$重合,连$TF$. \par
        $\because \Delta BAC$是等边三角形$\therefore \angle BAC=\angle ABC=\angle ACB=60^{\circ}$,即旋转角为$60^{\circ}$$\therefore \angle TAP=60^{\circ}$ \par
        由旋转的性质,知$\Delta TAB \text{\scalebox{-1}[1]{$\cong$}} \Delta PAC$,即$\angle TBA=\angle PCA$、$TB=PC$ \point{3} \par
        记$\angle TBA=\alpha$、$\angle ABF=\beta$,则$\angle TBF=\alpha+\beta$、$\angle PCA=\angle TBA=\alpha$、$\angle EBC=\angle ABC-\angle ABE=60^{\circ}-\beta$. \par
        $\because 菱形CPEQ,且\angle PCQ=60^{\circ}$,$\therefore PE=PC=TB$、$\angle PEQ=60^{\circ}$、$\angle PQC=120^{\circ}$$\therefore \angle ACQ=60^{\circ}-\alpha$ \par
        $\because 在四边形BEQC中,\angle EBC + \angle BCQ + \angle Q + \angle QEB = 360^{\circ} \therefore \angle BEP = \angle BEQ - \angle PEQ = \alpha + \beta = \angle TBF$ \point{4} \par
        又$\because F$为$BE$中点,即$BF=EF$$\therefore \Delta TBF \text{\scalebox{-1}[1]{$\cong$}} \Delta PEF \ (SAS) \ \therefore \angle TFB = \angle PFE、TF=PF \ \therefore T$、$F$、$P$三点共线 \point{5} \par
        又$\because TA=TP$、$\angle TAP=60^{\circ}$$\therefore \angle PAF=30^{\circ}$、$AF \bot PF$$\therefore AP=2PF$$\therefore AF=\sqrt{AP^2-PF^2}=\sqrt{3}PF$ \par
        综上所述,$AF \bot PF$ \point{6} 且$AF=\sqrt{3}PF$ \point{7}.
        \begin{figure}[!htb]
            \raggedleft
            \begin{tikzpicture}[scale=0.45]
            \coordinate[label=above:{$A$}] (A) at (3.49,6.05);
            \coordinate[label=below left:{$B$}] (B) at (0,0);
            \coordinate[label=below right:{$C$}] (C) at (6.98,0);
            \draw (A) -- (B) -- (C) -- cycle;
            \coordinate[label=below left:{$P$}] (P) at (5.48,1.51);
            \coordinate[label=right:{$Q$}] (Q) at (7.54,2.06);
            \coordinate[label=above:{$E$}] (E) at (6.03,3.57);
            \coordinate[label=below:{$F$}] (F) at (3.02,1.79);
            \draw (P) -- (C) -- (Q) -- (E) -- cycle;
            \draw (B) -- (E);
            \draw (A) -- (F) -- (P);
            \coordinate[label=left:{$T$}] (T) at (0.55,2.06);
            \draw[ansline] (B) -- (T) -- (A) -- (P);
            \draw[ansline] (F) -- (T);
        \end{tikzpicture}
        \end{figure}
        \subquestion $\frac{7}{4}\sqrt{2}+\frac{7}{4}\sqrt{6}-\frac{\sqrt{3}}{2}$ \point{10}
    \end{subquestions}
    \question %24
    \begin{subquestions}
        \subquestion $A(-1,0)$、$B(3,0)$、$C(0,-3)$、$D(1,-4)$ \point{2}
        \subquestion 设$CB:y=kx+b$,则
        $\begin{cases}
            3k+b = 0 \\
            b = -3
        \end{cases}$,解得
        $\begin{cases}
            k=1 \\
            b=-3
        \end{cases}$,即$CB:y=x-3$ \point{3} \par
        $\because EF:x=t$、$MN:x=t+1$$\therefore E(t,t-3)$、$F(t,t^2-2t-3)$、$M(t+1,t-2)$、$N(t+1,t^2-4)$ \par
        $\therefore EF=y_E-y_F=(t-3)-(t^2-2t-3)=-t^2+3t$、$MN=y_M-y_N=(t-2)-(t^2-4)=-t^2+t+2$,于是
        $$EF-MN=(-t^2+3t)-(-t^2+t+2)=2t-2 \point{4}$$
        又$\because $点$E$、$M$在线段$BC$(含两端)上 $\therefore 0 \leq t \leq 2$ \par
        $\begin{aligned}
            \quad \ \therefore \ & 当 0 \leq t < 1 时,EF<MN \point{5} \\
                       & 当 t = 1 时,EF=MN \point{6} \\
                       & 当 1 < t \leq 2 时,EF>MN \point{7}
        \end{aligned}$
        \begin{figure}[!htb]
            \raggedleft
            \begin{tikzpicture}[>=Stealth,scale=0.7]
                \draw[->] (-2,0) -- (5,0) node[below] {$x$};
                \draw[->] (0,-5) -- (0,2) node[right] {$y$};
                \coordinate[label=below right:{$O$}] (O) at (0,0);
                \draw (-1.24,1) parabola bend (1,-4) (3.24,1);
                \coordinate[label=below left:{$A$}] (A) at (-1,0);
                \coordinate[label=below right:{$B$}] (B) at (3,0);
                \coordinate[label=left:{$C$}] (C) at (0,-3);
                \draw (B) -- (C);
                \draw (0.8,-5) -- (0.8,2);
                \coordinate[label=left:{$E$}] (E) at (0.8,-2.2);
                \coordinate[label=left:{$F$}] (F) at (0.8,-3.96);
                \draw (1.8,-5) -- (1.8,2);
                \coordinate[label=left:{$M$}] (M) at (1.8,-1.2);
                \coordinate[label=left:{$N$}] (N) at (1.8,-3.36);
            \end{tikzpicture}
        \end{figure}
        \subquestion 设点$P(p,p^2-2p-3)$、$Q(q,q^2-2q-3)$、$PQ:y=k_1x+b_1$,则
        $\begin{cases}
            p^2-2p-3=k_1p+b_1 \\
            q^2-2q-3=k_1p+b_1 \\
        \end{cases}$,解得
        $\begin{cases}
            k_1=p+q-2 \\
            b_1=-pq-3
        \end{cases}$,即$PQ:y=(p+q-2)x-(pq+3)$ \par
        $\because PQ$过点$(2,-2)$,$\therefore 2(p+q-2)-(pq+3)=-2$,即
        $$pq=2p+2q-5 \ \ \ \ \ (\ast) \point{8} $$
        设$PB:y=k_2x+b_2$,则
        $\begin{cases}
            0 = 3k_2+b_2 \\
            p^2-2p-3 = k_2p+b_2
        \end{cases}$,解得
        $\begin{cases}
            k_2=p+1 \\
            b_2=-3p-3
        \end{cases}$,即$PB:y=(p+1)x-3(p+1)$ \par
        设$DQ:y=k_3x+b_3$,则
        $\begin{cases}
            -4 = k_3+b_3 \\
            q^2-2q-3 = k_3q+b_3
        \end{cases}$,解得
        $\begin{cases}
            k_3 = q-1 \\
            b_3 = -q-3
        \end{cases}$,即$DQ:y=(q-1)x-(q+3)$ \par
        联立$PB$和$DQ$的方程,有
        $\begin{cases}
            y=(p+1)x-3(p+1) \\
            y=(q-1)x-(q+3)
        \end{cases}$,解得
        $\begin{cases}
            x=\frac{3p-q}{p-q+2} \\
            y=\frac{2pq-6p+2q-6}{p-q+2}
        \end{cases}$ \par
        将$(\ast)$代入,得到
        $$T\left(\frac{3p-q}{p-q+2},\frac{-2p+6q-16}{p-q+2}\right) \point{10} $$
        猜想点$T$在一条定直线上运动,设这条直线为$l:y=kx+b$,则
        $$\begin{aligned}
            \frac{-2p+6q-16}{p-q+2} &= k \cdot \frac{3p-q}{p-q+2}+b \\
            -2p+6q-16 &= k(3p-q) + b(p-q+2) \\
            -2p+6q-16 &= 3kp-kq+bp-bq+2b \\
            (3k+b+2)p-(k+b+6)q+(2b+16) &= 0
        \end{aligned}$$
        若猜想为真,则上式在$p$、$q$变化时恒成立,即
        $\begin{cases}
            3k+b+2 = 0 \\
            k+b+6 = 0 \\
            2b+16 = 0
        \end{cases}$,
        这个方程组有解,为
        $\begin{cases}
            k=2 \\
            b=-8
        \end{cases}$,这说明点$T$始终在$l:y=2x-8$上运动. \point{11} \par
        作直线$l$交$x$轴于点$E$,交$y$轴于点$F$,则$E(4,0)$、$F(0,-8)$,于是$AE=5$、$OF=8$、$OE=4$、$EF=\sqrt{OE^2+OF^2}=4\sqrt{5}$.连$AF$,过$A$作$AT' \bot l$于点$T'$,则$AT \geq AT'$,即线段$AT$的长度最短为$AT'$. \par
        又$\because S_{\Delta AEF}=\frac{1}{2}AE\cdot OF=\frac{1}{2}EF\cdot AT' \therefore AT'=\frac{AE \cdot OF}{EF}=2\sqrt{5}$ \par
        综上所述,线段$AT$长度最小为$2\sqrt{5}$. \point{12}
        \begin{figure}[!htb]
            \raggedleft
            \begin{tikzpicture}[>=Stealth,scale=0.7]
                \draw[->] (-2,0) -- (5,0) node[below] {$x$};
                \draw[->] (0,-9) -- (0,2) node[right] {$y$};
                \coordinate[label=below right:{$O$}] (O) at (0,0);
                \draw (-1.24,1) parabola bend (1,-4) (3.24,1);
                \coordinate[label=below left:{$A$}] (A) at (-1,0);
                \coordinate[label=below right:{$B$}] (B) at (3,0);
                \coordinate[label=below:{$D$}] (D) at (1,-4);
                \draw (-1.52,-1.22) -- (4.46,-2.54);
                \coordinate[label=below:{$P$}] (P) at (-0.60,-1.43);
                \coordinate[label=below right:{$Q$}] (Q) at (2.38,-2.08);
                \draw (-1.51,-1.79) -- (4.80,0.71);
                \draw (0.54,-4.64) -- (4.76,1.20);
                \coordinate[label=below right:{$T$}] (T) at (4.25,0.49);
                \draw (A) -- (T);
                \draw[ansline] (-0.25,-8.5) -- (4.5,1);
                \coordinate[label=below right:{$E$}] (E) at (4,0);
                \coordinate[label=right:{$F$}] (F) at (0,-8);
                \draw[ansline] (A) -- (F);
                \coordinate[label=below right:{$T'$}] (Tt) at (3,-2);
                \draw[ansline] (A) -- (Tt);
            \end{tikzpicture}
        \end{figure}
    \end{subquestions}
\end{questions}
\end{document}