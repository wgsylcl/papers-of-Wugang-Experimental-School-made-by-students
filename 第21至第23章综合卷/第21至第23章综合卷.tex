\documentclass[10pt]{article}
\usepackage{mathpaper}

\begin{document}
\showsecret
\papertitle{第21$\boldsymbol{\sim}$23章综合卷}
\paperinformation{时间:2小时~~~~满分:120分}
\informationline
\begin{questions}{\selectingintroduction}
    \question 已知$a$是一个实数,则下列关于$x$的函数中,(~~~~~~~)一定是二次函数.
    \twp{$y=ax^2-x+1$}{$y=(x+1)(x-3)-(x+4)(x-8)$}{$y=(a^2-a+1)x^2+ax-a^2$}{$y=\left|a+1\right|x^2-7x+10a$}
    \question 一元二次方程$-3x^2-3x+5=0$的两根之和为(~~~~~~~).
    \onp{$-1$}{$1$}{$5$}{$-5$}
    \question %3
    \question %4
    \question 已知抛物线$y=ax^2+ax-b \ (a < 0)$上有三点$(1,y_1)$、$(-1,y_2)$、$(-3,y_3)$,则$y_1$、$y_2$、$y_3$之间的大小关系正确的是(~~~~~~~).
    \onp{$y_2>y_3>y_1$}{$y_2>y_1>y_3$}{$y_1>y_2>y_3$}{$y_1>y_3>y_2$}
    \question %6
    \question 一座桥的桥洞形状为抛物线,当水位正常时,水面的宽度是桥洞顶点到水面距离的2倍,水上涨3米后,水面的宽度是正常时的一半,则正常时水面宽(~~~~~~~)米.
    \onp{$3$}{$4$}{$6$}{$8$}
    \question 已知抛物线$C_1:y=-a(x-2)^2+b$过点$(1,4)$,则抛物线$C_2:y=a(x+1)^2-b+3$必过点(~~~~~~~).
    \onp{$(0,-1)$}{$(-2,1)$}{$(-4,7)$}{$(0,7)$}
    \question 已知两不等实数$m$、$n$满足$m^2-3m-1=0$、$n^2-3n-1=0$,则代数式$m^3-2m^2+n^2-\frac{1}{n}+2n$的值为(~~~~~~).
    \onp{$17$}{$-7$}{$-12$}{$12$}
    \question %10
\end{questions}

\begin{questions}{\complitingintroduction}
    \question 抛物线$y=2x^2+4x-1$的顶点是\complitingline
    \question %12
    \question 若当$-3 \leq x \leq 2$时,函数$y=x^2+4x+a$的最小值与最大值之积为$-28$,则$a$的值为\complitingline
    \question %14
    \question 已知实数$a$、$b$、$c$满足$a+b+c=0$、$0 < 2a \leq c$,则有下列说法:
    \begin{subsubquestions}
        \subsubquestion $9a+3b+c \geq 0$.
        \subsubquestion $5a+2b+c \leq 0$.
        \subsubquestion 对任意的$x \leq 2$,不等式$ax^2+bx+c \leq 4a+2b+c$恒成立.
        \subsubquestion 若$16a+4b+c=3$,则关于$x$的不等式$ax^2+(b-1)x+c+1 \geq 0$的解集是$1 \leq x \leq 4$.
    \end{subsubquestions}
    其中正确的是\complitingline
    \question %16
\end{questions}

\begin{questions}{\answeringintroduction}
    \question 分别求下列抛物线的开口方向、顶点坐标以及与$x$轴的交点坐标:
    \begin{subquestions}
        \subquestion $y=-2x^2+6x+8$
        \subquestion $y=x^2-8x-12$
    \end{subquestions}
    \question %18
    \begin{subquestions}
        \subquestion %18.1
        \subquestion %18.2
    \end{subquestions}
    \question %19
    \begin{subquestions}
        \subquestion %19.1
        \subquestion %19.2
    \end{subquestions}
    \question
    \begin{subquestions}
        \subquestion %20.1
        \subquestion %20.2
    \end{subquestions}
    \question %21
    \begin{subquestions}
        \subquestion %21.1
        \subquestion %21.2
    \end{subquestions}
    \question %22
    \begin{subquestions}
        \subquestion %22.1
        \subquestion %22.2
        \subquestion %22.3
    \end{subquestions}
    \question %23
    \begin{subquestions}
        \subquestion %23.1
        \subquestion %23.2
        \subquestion %23.3
    \end{subquestions}
    \question %24
    \begin{subquestions}
        \subquestion %24.1
        \subquestion %24.2
        \subquestion %24.3
    \end{subquestions}
\end{questions}
\end{document}