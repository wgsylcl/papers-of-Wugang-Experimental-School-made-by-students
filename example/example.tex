\documentclass[11pt]{article}
\usepackage[UTF8]{ctex}
\usepackage[a4paper,left=25mm,right=25mm,top=25mm]{geometry}
\usepackage{xeCJKfntef}\usepackage{amsmath, amsfonts,amssymb}   %数学符号
\newcommand{\onp} [4] { \\
    \begin{tabular} {*{4}{@{}p{3.5cm}}}
        A.~#1 & B.~#2 & C.~#3 & D.~#4
    \end{tabular}
}
\newcommand{\twp} [4] { \\
    \begin{tabular} {*{2}{@{}p{7cm}}}
        A.~#1 & B.~#2
    \end{tabular} \\
    \begin{tabular} {*{2}{@{}p{7cm}}}
        C.~#3 & D.~#4
    \end{tabular}
}
\newcommand{\fop} [4] { \\
    A.~#1 \\
    B.~#2 \\
    C.~#3 \\
    D.~#4
}
\newcommand{\smallpicture}[2]{\includegraphics[scale = #2]{#1}}
\usepackage{graphicx}  %插图包
\usepackage{paralist}  %序号包
\newcommand {\pbox} [3] {
    \unitlength=1mm
    \begin{picture} (0, 0)
        \put (#1, #2) {#3}
    \end{picture}
}
\begin{document}
\section*{\centering 试卷标题}
\section*{\normalsize 一、选择题(每小题3分)}
\begin{enumerate}\setcounter{enumi}{0}
    \item 设集合$U=\{1,2,3,4,5,6\}$,$A=\{1,3,6\}$,$B=\{2,3,4\}$,则$A\cap(\complement_UB)$=
    \onp{\{3\}}{\{1,6\}}{\{5,6\}}{\{1,3\}}
    \item 已知函数$f(x)$的定义域为$\mathbb{R}$,$f(x+2)$为偶函数,$f(2x+1)$为奇函数,则
    \twp{$f(-\frac{1}{2})=0$}{$f(-1)=0$}{$f(2)=0$}{$f(4)=0$}
    \item 某物理量的测量结果服从正态分布$N(10, \sigma^2)$,下列结论中不正确的是
    \fop{$\sigma$越小,该物理量在一次测量 中在(9.9, 10.1)的概率越大}
    {$\sigma$越小,该物理量在一次测量中大于10的概率为0.5}
    {$\sigma$越小,该物理量在一次测量中小于9.99与大于10.01的概率相等}
    {$\sigma$越小,该物理量在一次测量中落在(9.9, 10.2)与落在(10, 10.3)的概率相等}
    \item 在四棱锥$Q-ABCD$中,底面$ABCD$是正方形,若$AD=2$,$QD=QA=\sqrt{5}$,$QC=3$,
    \begin{compactenum}[(1)]
        \item 证明:平面$QAD$$\bot$平面$ABCD$;
        \item 求二面角$B-QD-A$的平面角的余弦值. \\
        \pbox {0}{-18} {\smallpicture{testgraph.png}{1}}
    \end{compactenum}
\end{enumerate}
\end{document}