\documentclass[10pt]{article}
\usepackage{mathpaper}
\begin{document}
\showsecret
\papertitle{第二十二章~~~二次函数}
\paperinformation{时间:2小时~~~~满分:120分}
\informationline
\begin{questions}{\selectingintroduction}
    \question 下列函数为二次函数的是(~~~~~~~)
    \twp{$y = x^{2} - 2x + 1 - (x - 1)(2x + 1)$}{$y = \frac{x^{4} - 3x^{3} + x^{2} + 1}{x}$}{$y = 18x - 16$}{$y = x^{2} + 2x + 3 - \frac{1}{x}$}
    \question 二次函数$y = x^{2} + ax + 1 - a$的图像必过点(~~~~~~~)
    \onp{$(0,1)$}{$(1,2)$}{$( - 1,0)$}{$(0,0)$}
    \question 已知二次函数$y = ax^{2} + bx + c$的顶点为$(4, - 2)$,且过点$(6,2)$,则(~~~~~~~)
    \twp{$a = - 1,b = 8,c = 18$}{$a = 1,b = 8,c = 14$}{$a = 1,b = - 8,c = 14$}{$a = - 1,b = 6,c = - 14$}
    \question 已知抛物线$y=ax^2-2ax+b \ (a < 0)$上有三点$A(2,y_1)$、$B(1,y_2)$、$C(-3,y_3)$,则$y_1$、$y_2$、$y_3$的大小关系正确的是(~~~~~~~)
    \onp{$y_3>y_2>y_1$}{$y_2>y_3>y_1$}{$y_2>y_1>y_3$}{$y_1>y_2>y_3$}
    \question 若二次函数$y = x^{2} + 4x + 4$的图像与一次函数$y = ax + a$的图像有且仅有一个交点,则(~~~~~~~)
    \onp{$a = 2$}{$a = 4$}{$a = - 1$}{$a = - 2$}
    \question 已知抛物线$y_1=ax^2+bx+c \ (a > 0)$与直线$y_2=mx+n \ (m \neq 0)$交于两点$A(1,8)$、$B(6,4)$,则不等式$y_1>y_2$的解集是(~~~~~~~)
    \onp{$4<x<8$}{$1<x<6$}{$x<4$或$x>8$}{$x<1$或$x>6$}
    \question 已知一本书的成本为$20$元,售价为$35$元,日销量$100$本,为了提高销售量,商家决定进行降价.经市场调研,每降价$1$元,日销量增加$2$本,则降低价格$x$(单位:元)与每日利润$y$(单位:元)之间的函数关系式正确的是(~~~~~~~)
    \twp{$y=(100+2x)(35-x)$}{$y=(100-2x)(15+x)$}{$y=(100-2x)(35+x)$}{$y=(100+2x)(15-x)$}
    \question 如图是一座桥的桥洞,已知此桥洞的形状是一条抛物线,且当水面宽$4$米时,水深$3$米;当水面宽$6$米时,水深$1.75$米,则桥洞顶点距离水底(~~~~~~~)米.
    \onp{$3$}{$4$}{$6$}{$8$}
    \begin{figure}[!htb]
        \raggedleft
        \subfigure[(第8题)]
        {
            \smallpicture{./images/bridge.png}{0.12}
        }
    \end{figure}
    \question 已知抛物线$y=ax^2-4ax-1$上有两点$A(x_1,y_1)$、$B(x_2,y_2)$满足$y_1<y_2$,则下列说法正确的是(~~~~~~~)
    \twp{若$x_1-x_2<0$,则$x_1+x_2-4<0$}{若$x_1-x_2<0$,则$x_1+x_2-4>0$}{若$x_1-x_2>0$,则$a(x_1+x_2-4)>0$}{若$x_1-x_2>0$,则$a(x_1+x_2-4)<0$}
    \question 关于$x$的方程$x^2-\left|x-2\right|-4x+k+1=0$不可能有(~~~~~~~)个不同的实数根.
    \onp{$1$}{$2$}{$3$}{$4$}
\end{questions}

\begin{questions}{\complitingintroduction}
    \question 抛物线$y = 4x^{2} + 6x + 2$与$x$轴的交点是\complitingline.
    \question 抛物线$y = x^{2} + 6x + 7$的顶点为\complitingline.
    \question 若将抛物线$y=x^2-2ax+a^2+2a$向右平移$5$个单位长度,再向下平移$5$个单位长度,得到的抛物线过原点,则$a$的值为\complitingline.
    \question 若抛物线$y = ax^{2} + bx + c$与抛物线$y = - a(x - 1)^{2} + 4a$关于$x$轴对称,则$\frac{c}{b}$的值为\complitingline.
    \question 如图,抛物线$y = ax^{2} + bx + c \ (a \neq 0)$的对称轴为直线$x = - 1$,则有下列说法:
    \begin{subsubquestions}
        \subsubquestion $abc < 0$;
        \subsubquestion $3a + c > 0$;
        \subsubquestion $\left( \frac{b}{a} \right)^{2} - \frac{4c}{a} > 4$;
        \subsubquestion 当抛物线经过点$\left( \frac{1}{2},2 \right)$时,若方程$ax^{2} + bx + c - 2 = 0$的两根为$x_{1},x_{2}\left( x_{1} < x_{2} \right)$,则$x_{1} + 2x_{2} = - \frac{3}{2}$;
        \subsubquestion 若在方程$\left| ax^{2} + bx + c \right| = k$中,$k$为常数,且$0 < k < - a + b - c$,则方程所有根的和为$- 4$;
    \end{subsubquestions}
    其中正确的有\complitingline.
    \begin{figure}[!htb]
        \centering
        \raggedleft
        \subfigure[(第15题)]{
        \begin{tikzpicture}[scale=0.35,>=Stealth]
            \draw[->] (-5.75,0) -- (2.2,0) node[below] {$x$};
            \draw[->] (0,-5.2) -- (0,5.2) node[right] {$y$};
            \draw (-1.6,0.1)--(-1.6,0) node[below] {$-1$};
            \draw (1.6,0.1)--(1.6,0) node[below] {$1$};
            \draw (-4.9333333333,5) parabola bend (-1.6,-5) (1.7333333333,5);
        \end{tikzpicture}}
    \end{figure}
    \question 若关于$x$的方程$ax^2-3x-1=0$的所有实根均满足$-1<x<0$,则$a$的取值范围是\complitingline.
\end{questions}

\begin{questions}{\answeringintroduction}
    \question 已知在平面直角坐标系内有一条抛物线过点$(-2,2)$、$(3,2)$和$(2,-4)$,求这条抛物线的顶点坐标.
    \addemptyline\addemptyline\addemptyline\addemptyline
    \question 在一次足球联赛中,共有$x$支球队参加,已知每两个队伍间都需进行一场比赛,且他们一共打了$y$场比赛.
    \begin{subquestions}
        \subquestion 求$y$与$x$之间的函数关系式,并写出$x$的取值范围.
        \subquestion 比赛场数可能为$18$吗?为什么?
    \end{subquestions}
    \addspace
    \question 体育课上,同学们正在打篮球,如图,甲同学在水平距离球筐$10$米的地方投出一个三分球.已知球在出手时距离地面$1.8$米,水平飞行$5.5$米后达到最高$4.825$米,且球筐高$2.8$米.现建立如图的坐标系.
    \begin{subquestions}
        \subquestion 试计算说明此球能不能被准确投入篮筐.
        \subquestion 现在乙同学在甲同学正前方准备盖帽,已知他跳起摸高最高为$2.8$米,则他最远离甲同学多少米才能使盖帽成功?(补充:篮球在下降阶段不可盖帽)
    \end{subquestions}
    \begin{figure}[!htb]
        \raggedleft
        \subfigure[(第19题)]{
        \begin{tikzpicture}[>=Stealth,scale=0.5235]
            \coordinate[label=below right:{$O$}] (O) at (0,0);
            \draw[->] (-1,0) -- (11,0) node[below] {$x$};
            \draw[->] (0,-1) -- (0,5) node[right] {$y$};
            \draw (10,0) -- (10,2.8);
            \draw (0,1.8) parabola bend (5.5,4.825) (10,2.8);
            \filldraw (0,1.8) circle (.1);
            \filldraw (10,2.8) circle (.1);
        \end{tikzpicture}}
    \end{figure}
    \newpage
    \question 已知二次函数$y=x^2+ax+2a$的图像与$x$轴有两个交点,且这两个交点间的距离为$3$.
    \begin{subquestions}
        \subquestion 求$a$.
        \subquestion 若此函数的图像交$y$轴于负半轴,直接写出当$b \le x \le b+5$时,函数值$y$的最小值.
    \end{subquestions}
    \addspace
    \question 如图,在平面直角坐标系中,顶点为点$(1,-1)$的一抛物线过原点.
    \begin{subquestions}
        \subquestion 直接写出这条抛物线的解析式.
        \subquestion 记这条抛物线与$x$轴的另一交点为$A$,过点$A$作直线$AB$交抛物线于第二象限一点$B$,使$\angle BAO=45^{\circ}$.已知点$E$是抛物线上线段$AB$下方一动点,过$E$作直线$l$垂直于$y$轴,交直线$AB$于一点$F$,求线段$EF$长度的最大值.
    \end{subquestions}
    \begin{figure}[!htb]
        \raggedleft
        \subfigure[(第21题)]{
            \begin{tikzpicture}[>=Stealth,scale=0.7]
                \draw[->] (-2,0) -- (4,0) node[below] {$x$};
                \draw[->] (0,-2) -- (0,4) node[right] {$y$};
                \coordinate[label=below left:{$O$}] (O) at (0,0);
                \draw (-1.12,3.5) parabola bend (1,-1) (3.12,3.5);
                \coordinate[label=below right:{$A$}] (A) at (2,0);
                \coordinate[label=below left:{$B$}] (B) at (-1,3);
                \coordinate[label=below left:{$E$}] (E) at (-0.41,1);
                \coordinate[label=below left:{$F$}] (F) at (1,1);
                \draw (-1.5,3.5) -- (3.5,-1.5);
                \draw (-1.5,1) -- (3.5,1);
            \end{tikzpicture}}
    \end{figure}
    \question 2015年初,草莓进入采摘旺季,某公司经营销售草莓的业务,以$3$万元/吨的价格向农户收购后,分拣成甲、乙两类,甲类草莓包装后直接销售,乙类草莓深加工后再销售.甲类草莓的包装成本为$1$万元/吨,且当甲类草莓的销售量$m$不超过$8$吨时,它的平均销售价格$n = - m + 14$,当甲类草莓的销售量超过$8$吨时,它的平均销售价格为$6$万元/吨;乙类草莓加工总费用$s$(单位:万元)与加工数量$t$(单位:吨)之间的函数关系为$s = 12 + 3t$,平均销售价格为$9$万元/吨. \\
    在该公司的某次收购中,计划甲类草莓分$x$吨、乙类草莓分$y$吨,假设经营这批草莓所获得的总利润为$w$万元.
    \begin{subquestions}
        \subquestion 求$w$与$x$和$y$之间的函数关系式.
        \subquestion 若该公司收购了$20$吨的草莓,且获得了$30$万元的总利润,求用于销售甲类的草莓有多少吨?
        \subquestion 若该公司准备投入$100$万元资金进行收购及后续加工,请你设计一种收购与经营方案,使该公司获得最大的总利润,并求出最大的总利润.
    \end{subquestions}
    \newpage
    \question 在平面直角坐标系中,直线$AB:y=-\frac{1}{2}x+3$与抛物线$y=\frac{1}{2}x^2$交于$A$、$B$两点,点$P$是抛物线上异于$O$的一动点.
    \begin{subquestions}
        \subquestion 求点$A$、$B$的坐标.
        \subquestion 如图1,求所有的点$P$,使$P$到直线$AB$的距离与$O$到直线$AB$的距离相等.
        \subquestion 如图2,$Q$是$y$轴上一动点,直接写出所有的组合$P$、$Q$,使得$A$、$B$、$P$、$Q$构成的四边形是平行四边形.
    \end{subquestions}
    \begin{figure}[!htb]
        \centering
        \subfigure[(1)]{
        \begin{tikzpicture}[>=Stealth,scale=0.7]
            \draw[->] (0,-1) -- (0,7) node[below right] {$y$};
            \draw[->] (-4,0) -- (4,0) node[below left] {$x$};
            \draw (-3.46410,6) parabola bend (0,0) (3.46410,6);
            \coordinate[label=below left:{$O$}] (O) at (0,0);
            \coordinate[label=below left:{$A$}] (A) at (-3,4.5);
            \coordinate[label=below right:{$B$}] (B) at (2,2);
            \coordinate[label=below left:{$P$}] (P) at (-1,0.5);
            \draw (A) -- (B) -- (P) -- cycle;
        \end{tikzpicture}}
        \qquad\qquad
        \subfigure[(2)]{
        \begin{tikzpicture}[>=Stealth,scale=0.7]
            \draw[->] (0,-1) -- (0,7) node[below right] {$y$};
            \draw[->] (-4,0) -- (4,0) node[below left] {$x$};
            \draw (-3.46410,6) parabola bend (0,0) (3.46410,6);
            \coordinate[label=below left:{$O$}] (O) at (0,0);
            \coordinate[label=below left:{$A$}] (A) at (-3,4.5);
            \coordinate[label=below right:{$B$}] (B) at (2,2);
            \coordinate[label=left:{$P$}] (P) at (-1,0.5);
            \coordinate[label=right:{$Q$}] (Q) at (0,6);
            \draw (A) -- (P) -- (B) -- (Q) -- cycle;
        \end{tikzpicture}}
        \caption*{(第23题)}
    \end{figure}
    \question 在平面直角坐标系中,抛物线$y=ax^2+x+c$的对称轴为$x=1$,且与$x$轴交于点$A(4,0)$和点$B$,与$y$轴交于点$C$.
    \begin{subquestions}
        \subquestion 求抛物线的解析式.
        \subquestion 如图1,连$BC$、$AC$,$D$是抛物线上一点,连$DC$,若$AC$平分$\angle BCD$,求点$D$的坐标.
        \subquestion 如图2,点$P$是直线$y=5$上、但不在抛物线对称轴上的动点,过点$P$且不与$y$轴平行的两条直线$l_1$、$l_2$与抛物线均只有一个交点,$l_1$、$l_2$分别交抛物线对称轴与点$M$、$N$,点$G$为抛物线对称轴上点$M$、$N$下方一点,若${GP}^2=GM \cdot GN$恒成立,求点$G$的坐标.
    \end{subquestions}
    \begin{figure}[htb]
        \centering
        \subfigure[(1)]{
            \begin{tikzpicture}[>=Stealth,scale=0.7]
                \draw[->] (-3,0)--(5,0) node[below] {$x$};
                \draw[->] (0,-1.75)--(0,6) node[right] {$y$};
                \coordinate[label=below left:{$O$}] (O) at (0,0);
                \coordinate[label=below left:{$A$}] (A) at (4,0);
                \coordinate[label=below right:{$B$}] (B) at (-2,0);
                \coordinate[label=above left:{$C$}] (C) at (0,4);
                \coordinate[label=above right:{$D$}] (D) at (1,4.5);
                \draw (A)--(C)--(B);
                \draw (C)--(D);
                \draw (-2.5,-1.625) parabola bend (1,4.5) (4.5,-1.625);
            \end{tikzpicture}
        }
        \qquad\qquad
        \subfigure[(2)]{
            \begin{tikzpicture}[>=Stealth,scale=0.7]
                \draw[->] (-3,0)--(5,0) node[below] {$x$};
                \draw[->] (0,-1.75)--(0,6) node[right] {$y$};
                \coordinate[label=below left:{$O$}] (O) at (0,0);
                \coordinate[label=below left:{$A$}] (A) at (4,0);
                \coordinate[label=below right:{$B$}] (B) at (-2,0);
                \coordinate[label=above left:{$C$}] (C) at (0,4);
                \coordinate[label=above left:{$M$}] (M) at (1,9.79);
                \coordinate[label=above right:{$P$}] (P) at (2.46,5);
                \coordinate[label=below:{$N$}] (N) at (1,4.56);
                \draw[densely dashed] (1,-1.7)--(1,10.75);
                \draw (P)--(-1.67,3.75);
                \draw (M)--(4.23,-0.72);
                \draw (-3,5)--(5,5);
                \draw (-2.5,-1.625) parabola bend (1,4.5) (4.5,-1.625);
            \end{tikzpicture}
        }
        \caption*{(第24题)}
    \end{figure}
\end{questions}
\end{document}