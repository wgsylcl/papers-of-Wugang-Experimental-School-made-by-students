\documentclass[10pt]{article}
\usepackage{mathpaper}
% setpreview
\begin{document}
\showsecret
\papertitle{第二十二章~~~二次函数}
\paperinformation{时间:2小时~~~~满分:120分}
\informationline
\begin{questions}{\selectingintroduction}
    \question 下列函数为二次函数的是(~~~~~~~)
    \twp{$y = x^{2} - 2x + 1 - (x - 1)(2x + 1)$}{$y = \frac{x^{4} - 3x^{3} + x^{2} + 1}{x}$}{$y = 18x - 16$}{$y = x^{2} + 2x + 3 - \frac{1}{x}$}
    \question 二次函数$y = x^{2} + ax + 1 - a$的图像必过点(~~~~~~~)
    \onp{$(0,1)$}{$(1,2)$}{$( - 1,0)$}{$(0,0)$}
    \question 若二次函数$y = x^{2} + 4x + 4$的图像与一次函数$y = ax + a$的图像有且仅有一个交点,则(~~~~~~~)
    \onp{$a = 2$}{$a = 4$}{$a = - 1$}{$a = - 2$}
    \question 已知二次函数$y = ax^{2} + bx + c$的顶点为$(4, - 2)$,且过点$(6,2)$,则(~~~~~~~)
    \twp{$a = - 1,b = 8,c = 18$}{$a = 1,b = 8,c = 14$}{$a = 1,b = - 8,c = 14$}{$a = - 1,b = 6,c = - 14$}
    \question 5
    \question 6
    \question 7
    \question 8
    \question 若关于$x$的一元二次方程$ax^2-2ax-2=0$在$-1<x<4$范围内有且仅有一根,则实数$a$的取值范围是(~~~~~~~)
    \onp{$a > \frac{2}{3}$}{$a < \frac{1}{4}$}{$\frac{1}{4} < a < \frac{2}{3}$}{$a > \frac{2}{3}$或$a < \frac{1}{4}$}
    % \question 如图,在平行四边形$ABCD$中,$BC=2AB=20$,$B$、$F$、$E$三点共线,且$\angle ABC=60^{\circ}$.连$AF$、$AE$、$CE$,若$AE=EF$,$\angle DAE+\angle CBF=60^{\circ}$,且$AF=6$,则$\Delta BEC$的面积为(~~~~~~~)
    % \onp{$16\sqrt{3}$}{$68\sqrt{3}$}{$32\sqrt{3}$}{$34\sqrt{3}$}
    % \begin{figure}[htb]
    %     \centering
    %     \raggedleft
    %     \subfigure[(第10题)]{
    %     \begin{tikzpicture}[scale=0.4]
    %         \coordinate[label=above:{$A$}] (A) at (2.5,4.33333);
    %         \coordinate[label=left:{$B$}] (B) at (0,0);
    %         \coordinate[label=above:{$D$}] (D) at (12.5,4.33333);
    %         \coordinate[label=right:{$C$}] (C) at (10,0);
    %         \coordinate[label=above right:{$E$}] (E) at (4.76,3.24);
    %         \coordinate[label=below right:{$F$}] (F) at (2.69,1.83);
    %         \draw (A)--(B)--(C)--(D)--cycle;
    %         \draw (A)--(F);
    %         \draw (A)--(E);
    %         \draw (C)--(E);
    %         \draw (B)--(E);
    %     \end{tikzpicture}}
    % \end{figure}
    \question 已知$x$、$y$为实数,则代数式$\sqrt{x^2+y^2}+\sqrt{x^2-4x+4+y^2}+\sqrt{x^2+y^2-4y+4}$的最小值为(~~~~~~~).
    \onp{$4$}{$3\sqrt{2}$}{$\sqrt{6}+\sqrt{2}$}{$2\sqrt{3}$}
\end{questions}

\begin{questions}{\complitingintroduction}
    \question 抛物线$y = 4x^{2} + 6x + 2$与$x$轴的交点是\complitingline\complitingline.
    \question 二次函数$y = x^{2} + 6x + 7$的顶点为\complitingline.
    \question 已知实数$a$、$b$、$c$满足$a \neq 0$,且$a - b + c = 0$、$9a + 3b + c = 0$,则抛物线$y = ax^{2} + bx + c$图像上一点$A( - 1,3)$关于抛物线对称轴对称的点为\complitingline.
    \question 把二次函数$y = ax^{2} + bx + c\ (a > 0)$的图像作关于$x$轴的对称变换,所得图像的解析式为$y = - a(x - 1)^{2} + 4a$,若$(m - 1)a + b + c \leq 0$,则$m$的最大值为\complitingline.
    \question 如图,抛物线$y = ax^{2} + bx + c (a \neq 0)$的对称轴为直线$x = - 1$,则有下列说法:
    \begin{subsubquestions}
        \subsubquestion $abc < 0$;
        \subsubquestion $3a + c > 0$;
        \subsubquestion $\left( \frac{b}{a} \right)^{2} - \frac{4c}{a} > 4$;
        \subsubquestion 当抛物线经过点$\left( \frac{1}{2},2 \right)$时,若方程$ax^{2} + bx + c - 2 = 0$的两根为$x_{1},x_{2}\left( x_{1} < x_{2} \right)$,则$x_{1} + x_{2} = - \frac{3}{2}$;
        \subsubquestion 若在方程$\left| ax^{2} + bx + c \right| = k$中,$k$为常数,且$0 < k < - a + b - c$,则方程所有根的和为$- 4$;
    \end{subsubquestions}
    其中正确的有\complitingline\complitingline.
    \begin{figure}[!htb]
        \centering
        \raggedleft
        \subfigure[(第15题)]{
        \begin{tikzpicture}[scale=0.35,>=Stealth]
            \draw[->] (-5.75,0) -- (2.2,0) node[below] {$x$};
            \draw[->] (0,-5.2) -- (0,5.2) node[right] {$y$};
            \draw (-1.6,0.1)--(-1.6,0) node[below] {$-1$};
            \draw (1.6,0.1)--(1.6,0) node[below] {$1$};
            \draw (-4.9333333333,5) parabola bend (-1.6,-5) (1.7333333333,5);
        \end{tikzpicture}}
    \end{figure}
    \question 若关于$x$的方程$ax^2-3x-1=0$的所有实根均满足$-1<x<0$,则$a$的取值范围是\complitingline.
\end{questions}

\begin{questions}{\answeringintroduction}
    \question 已知在平面直角坐标系内有一条抛物线过点$(-2,2)$、$(3,2)$和$(2,-4)$,求这条抛物线的顶点坐标.
    \question 如图,在平面直角坐标系中,顶点为点$(1,-1)$的一抛物线过原点.
    \begin{subquestions}
        \subquestion 求这条抛物线的解析式.
        \subquestion 记这条抛物线与$x$轴的另一交点为$A$,过点$A$作直线$AB$交抛物线于第二象限一点$B$,使$\angle BAO=45^{\circ}$.已知点$E$是抛物线上线段$AB$下方一动点,过$E$作直线$l$垂直于$y$轴,交直线$AB$于一点$F$,求线段$EF$长度的最小值.
    \end{subquestions}
    \begin{figure}[!htb]
        \centering
            \begin{tikzpicture}[>=Stealth,scale=0.7]
                \draw[->] (-2,0) -- (4,0) node[below] {$x$};
                \draw[->] (0,-2) -- (0,4) node[right] {$y$};
                \coordinate[label=below left:{$O$}] (O) at (0,0);
                \draw (-1.12,3.5) parabola bend (1,-1) (3.12,3.5);
                \coordinate[label=below right:{$A$}] (A) at (2,0);
                \coordinate[label=below left:{$B$}] (B) at (-1,3);
                \coordinate[label=below left:{$E$}] (E) at (-0.41,1);
                \coordinate[label=below left:{$F$}] (F) at (1,1);
                \draw (-1.5,3.5) -- (3.5,-1.5);
                \draw (-1.5,1) -- (3.5,1);
            \end{tikzpicture}
        \caption*{(第18题)}
    \end{figure}
    \question 已知二次函数$y=x^2+ax+2a$的图像与$x$轴有两个交点,且这两个交点间的距离为$3$.
    \begin{subquestions}
        \subquestion 求$a$.
        \subquestion 试讨论当$b \le x \le b+5$时,函数值$y$的最小值.
    \end{subquestions}
    \question 20
    \begin{subquestions}
        \subquestion 20.1
        \subquestion 20.2
    \end{subquestions}
    \question 在平面直角坐标系中,抛物线$y = - \frac{1}{2}x^{2} - \frac{3}{2}x + c$交$x$轴于$A$、$B$两点,交$y$轴正半轴于点$C$,顶点为点$D$.
    \begin{subquestions}
        \subquestion 如图1,当$c = 2$时,若$P$为第四象限抛物线上一点,使得$\angle CAP = 45^{\circ}$,求点$P$的坐标.
        \subquestion 如图2,过$O$作$MN$平行于$CD$分别交抛物线于$M$、$N$,若$MN = 4CD$,求$c$的值.
    \end{subquestions}
    \begin{figure}[!htb]
        \centering
        \subfigure[(1)]
        {\begin{tikzpicture}[>=Stealth,scale=0.7]
            \draw[->] (0,-3) -- (0,4) node[below right] {$y$};
            \draw[->] (-5,0) -- (2,0) node[below left] {$x$};
            \draw (-4.93,-2.75) parabola bend (-1.5,3.125) (1.93,-2.75);
            \coordinate[label=below left:{$O$}] (O) at (0,0);
            \coordinate[label=above left:{$A$}] (A) at (-4,0);
            \coordinate[label=below left:{$B$}] (B) at (1,0);
            \coordinate[label=right:{$C$}] (C) at (0,2);
            \coordinate[label=right:{$P$}] (P) at (1.67,-1.89);
            \draw (A) -- (P);
            \draw (A) -- (C);
        \end{tikzpicture}}
        \subfigure[(2)]
        {
            \begin{tikzpicture}[>=Stealth,scale=0.5235]
                \draw[->] (0,-3.5) -- (0,6) node[below right] {$y$};
                \draw[->] (-6,0) -- (3,0) node[below left] {$x$};
                \draw (-5.59,-3) parabola bend (-1.5,5.34) (2.59,-3);
                \coordinate[label=below left:{$O$}] (O) at (0,0);
                \coordinate[label=below right:{$A$}] (A) at (-4.77,0);
                \coordinate[label=below left:{$B$}] (B) at (1.77,0);
                \coordinate[label=right:{$C$}] (C) at (0,4.22);
                \coordinate[label=above:{$D$}] (D) at (-1.5,5.34);
                \coordinate[label=left:{$M$}] (M) at (-3.75,2.81);
                \coordinate[label=right:{$N$}] (N) at (2.25,-1.69);
                \draw (C) -- (D);
                \draw (M) -- (N);
            \end{tikzpicture}
        }
        \caption*{(第21题)}
    \end{figure}
    \question 年初,草莓进入采摘旺季,某公司经营销售草莓的业务,以$3$万元/吨的价格向农户收购后,分拣成甲、乙两类,甲类草莓包装后直接销售,乙类草莓深加工后再销售.甲类草莓的包装成本为$1$万元/吨,且当甲类草莓的销售量$x$不超过$8$吨时,它的平均销售价格$y = - x + 14$, 当甲类草莓的销售量超过$8$吨时,它的平均销售价格为$6$万元/吨;乙类草莓加工总费用$s$(单位:万元)与加工数量$t$(单位:吨)之间的函数关系为$s = 12 + 3t$,平均销售价格为$9$万元/吨.
    \begin{subquestions}
        \subquestion 某次该公司收购了$20$吨的草莓,其中甲类草莓有$x$吨,经营这批草莓所获得的总利润为$w$万元.
        \begin{subsubquestions}
            \subsubquestion 求$w$与$x$之间的函数关系式.
            \subsubquestion 若该公司获得了$30$万元的总利润,求用于销售甲类的草莓有多少吨?
        \end{subsubquestions}
        \subquestion 在某次收购中,该公司准备投入$100$万元资金,请你设计一种经营方案,使该公司获得最大的总利润,并求出最大的总利润.
    \end{subquestions}
    \newpage
    \question 在平面直角坐标系中,直线$AB:y=-\frac{1}{2}x+3$与抛物线$y=\frac{1}{2}x^2$交于$A$、$B$两点,点$P$是抛物线上一动点.
    \begin{subquestions}
        \subquestion 求点$A$、$B$的坐标.
        \subquestion 如图1,当点$P$在直线$AB$下方时,求所有的点$P$,使$\Delta APB$的面积为$5$.
        \subquestion 如图2,$Q$是$y$轴上一动点,直接写出所有的组合$P$、$Q$,使得$A$、$B$、$P$、$Q$构成的四边形是平行四边形.
    \end{subquestions}
    \begin{figure}[!htb]
        \centering
        \subfigure[(1)]{
        \begin{tikzpicture}[>=Stealth,scale=0.7]
            \draw[->] (0,-1) -- (0,7) node[below right] {$y$};
            \draw[->] (-4,0) -- (4,0) node[below left] {$x$};
            \draw (-3.46410,6) parabola bend (0,0) (3.46410,6);
            \coordinate[label=below left:{$O$}] (O) at (0,0);
            \coordinate[label=below left:{$A$}] (A) at (-3,4.5);
            \coordinate[label=below right:{$B$}] (B) at (2,2);
            \coordinate[label=below left:{$P$}] (P) at (-1.33,0.89);
            \draw (A) -- (B) -- (P) -- cycle;
        \end{tikzpicture}}
        \qquad\qquad
        \subfigure[(2)]{
        \begin{tikzpicture}[>=Stealth,scale=0.7]
            \draw[->] (0,-1) -- (0,7) node[below right] {$y$};
            \draw[->] (-4,0) -- (4,0) node[below left] {$x$};
            \draw (-3.46410,6) parabola bend (0,0) (3.46410,6);
            \coordinate[label=below left:{$O$}] (O) at (0,0);
            \coordinate[label=below left:{$A$}] (A) at (-3,4.5);
            \coordinate[label=below right:{$B$}] (B) at (2,2);
            \coordinate[label=left:{$P$}] (P) at (-1,0.5);
            \coordinate[label=right:{$Q$}] (Q) at (0,6);
            \draw (A) -- (P) -- (B) -- (Q) -- cycle;
        \end{tikzpicture}}
        \caption*{(第23题)}
    \end{figure}
    \question 在平面直角坐标系中,抛物线$y=ax^2+bx+c$的对称轴为$x=1$,且与$x$轴交于点$A(4,0)$和点$B$,与$y$轴交于点$C$.
    \begin{subquestions}
        \subquestion 求抛物线的解析式.
        \subquestion 如图1,连$BC$、$AC$,$D$是抛物线上一点,连$DC$,若$AC$平分$\angle BCD$,求点$D$的坐标.
        \subquestion 如图2,点$P$是直线$y=5$上、但不在抛物线对称轴上的动点,过点$P$且不与$y$轴平行的两条直线$l_1$、$l_2$与抛物线均只有一个交点,$l_1$、$l_2$分别交抛物线对称轴与点$M$、$N$,点$G$为抛物线对称轴上点$M$、$N$下方一点,若${GP}^2=GM \cdot GN$恒成立,求点$G$的坐标.
    \end{subquestions}
    \begin{figure}[htb]
        \centering
        \subfigure[(1)]{
            \begin{tikzpicture}[>=Stealth,scale=0.7]
                \draw[->] (-3,0)--(5,0) node[below] {$x$};
                \draw[->] (0,-1.75)--(0,6) node[right] {$y$};
                \coordinate[label=below left:{$O$}] (O) at (0,0);
                \coordinate[label=below left:{$A$}] (A) at (4,0);
                \coordinate[label=below right:{$B$}] (B) at (-2,0);
                \coordinate[label=above left:{$C$}] (C) at (0,4);
                \coordinate[label=above right:{$D$}] (D) at (1,4.5);
                \draw (A)--(C)--(B);
                \draw (C)--(D);
                \draw (-2.5,-1.625) parabola bend (1,4.5) (4.5,-1.625);
            \end{tikzpicture}
        }
        \qquad\qquad
        \subfigure[(2)]{
            \begin{tikzpicture}[>=Stealth,scale=0.7]
                \draw[->] (-3,0)--(5,0) node[below] {$x$};
                \draw[->] (0,-1.75)--(0,6) node[right] {$y$};
                \coordinate[label=below left:{$O$}] (O) at (0,0);
                \coordinate[label=below left:{$A$}] (A) at (4,0);
                \coordinate[label=below right:{$B$}] (B) at (-2,0);
                \coordinate[label=above left:{$C$}] (C) at (0,4);
                \coordinate[label=above left:{$M$}] (M) at (1,9.79);
                \coordinate[label=above right:{$P$}] (P) at (2.46,5);
                \coordinate[label=below:{$N$}] (N) at (1,4.56);
                \draw[densely dashed] (1,-1.7)--(1,10.75);
                \draw (P)--(-1.67,3.75);
                \draw (M)--(4.23,-0.72);
                \draw (-3,5)--(5,5);
                \draw (-2.5,-1.625) parabola bend (1,4.5) (4.5,-1.625);
            \end{tikzpicture}
        }
        \caption*{(第24题)}
    \end{figure}
    % 解:
    % \\
    % 因为$P$在$y = 5$上,因此可设$P(p,5)$,又设直线$l:y = kx + b$过点$P$,于是有
    % \\
    % $$5 = kp + b$$
    % \\
    % 于是
    % \\
    % $$l:y = kx - kp + 5$$
    % \\
    % 联立$l$与抛物线的方程,得到
    % \\
    % $$kx - kp + 5 = - \frac{1}{2}x^{2} + x + 2$$
    % \\
    % 即
    % \\
    % $$x^{2} + 2(k - 1)x - 2(kp - 3) = 0$$
    % \\
    % 当$l$与抛物线有且仅有一个交点时,上方程仅有一根,则
    % \\
    % $$\Delta = B^{2} - 4AC = 4(k - 1)^{2} + 8(kp - 3) = 0$$
    % \\
    % 化简得到
    % \\
    % $$k^{2} + 2(p - 1)k - 5 = 0$$
    % \\
    % 由韦达定理,得
    % \\
    % $$\left\{ \begin{aligned}
    % & k_{1} + k_{2} = 2(1 - p) \\
    % & k_{1}k_{2} = - 5 \\
    % \end{aligned} \right. $$
    % \\
    % 于是可设
    % \\
    % $$PC:y = k_{1}x - k_{1}p + 5,\ PA:y = k_{2}x - k_{2}p + 5$$
    % \\
    % 令$x = 1$,得
    % \\
    % $$N\left( 1,k_{1} - k_{1}p + 5 \right),M(1,k_{2} - k_{2}p + 5)$$
    % \\
    % 设点$G(1,g)$,由$GP^{2} = GM \cdot GN$可知
    % \\
    % $$\begin{align}
    %     (p - 1)^{2} + (g - 5)^{2} &= \left( k_{1} - k_{1}p + 5 - g \right)\left( k_{2} - k_{2}p + 5 - g \right)\\
    %     &= \left( k_{1} - k_{1}p \right)\left( k_{2} - k_{2}p \right) + (g - 5)^{2} + \left( k_{1} - k_{1}p + k_{2} - k_{2}p \right)(5 - g)\\
    %     &= k_{1}k_{2}(1 - p)^{2} + (g - 5)^{2} + \left( k_{1} + k_{2} \right)(1 - p)(5 - g)
    % \end{align}$$
    % \\
    % 即
    % \\
    % $$(p - 1)^{2} = k_{1}k_{2}(1 - p)^{2} + \left( k_{1} + k_{2} \right)(1 - p)(5 - g)$$
    % \\
    % 代入的$k_1+k_2$和$k_1k_2$的值,得
    % \\
    % $$(p - 1)^{2} = - 5(p - 1)^{2} + 2(1 - p)^{2}(5 - g)$$
    % \\
    % 化简,得
    % \\
    % $$(g - 2)(p - 1)^{2} = 0$$
    % \\
    % 由于当$p$变化时,上式恒成立,故$g = 2$,即$G(1,2)$
    % \\
    % 综上所述,$G(1,2)$.
\end{questions}
\end{document}