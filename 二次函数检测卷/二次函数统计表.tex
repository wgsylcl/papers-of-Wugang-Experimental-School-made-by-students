\documentclass[10pt]{article}
\usepackage{mathpaper}
\usepackage{tabularx}

\begin{document}
\papertitle{试卷统计表}
\begin{table}[!htbp]
    \caption{总览}
    \centering
    \begin{tabularx}{0.9\textwidth}{|*{4}{>{\centering\arraybackslash}X|}} \hline
        试卷名称 & \qquad & 考察范围 & \qquad \\ \hline
        预计考生人数 & \qquad & 实际考生人数 & \qquad \\ \hline
        预计测试时间 & \qquad & 实际测试时间 & \qquad \\ \hline
        试卷总分 & \qquad & 预计试卷平均分 & \qquad \\ \hline
    \end{tabularx}
\end{table}
\begin{table}[!htbp]
    \caption{整体情况}
    \centering
    \begin{tabularx}{0.5\textwidth}{|*{3}{>{\centering\arraybackslash}X|}} \hline
        分数段 & 预计段内人数 & 实际段内人数 \\ \hline
        $(115,120]$ & \qquad & \qquad \\ \hline
        $(110,115]$ & \qquad & \qquad \\ \hline
        $(100,110]$ & \qquad & \qquad \\ \hline
        $(90,100]$ & \qquad & \qquad \\ \hline
        $(72,90]$ & \qquad & \qquad \\ \hline
        $[0,72]$ & \qquad & \qquad \\ \hline
    \end{tabularx}
\end{table}
\begin{table}[!htbp]
    \caption{非解答题情况}
    \centering
    \begin{tabularx}{\textwidth}{|*{6}{>{\centering\arraybackslash}X|}} \hline
    题号 & 预计得分人数 & 实际得分人数 & 题号 & 预计得分人数 & 实际得分人数 \\ \hline
    1 & \qquad & \qquad & 2 & \qquad & \qquad \\ \hline
    3 & \qquad & \qquad & 4 & \qquad & \qquad \\ \hline
    5 & \qquad & \qquad & 6 & \qquad & \qquad \\ \hline
    7 & \qquad & \qquad & 8 & \qquad & \qquad \\ \hline
    9 & \qquad & \qquad & 10 & \qquad & \qquad \\ \hline
    11 & \qquad & \qquad & 12 & \qquad & \qquad \\ \hline
    13 & \qquad & \qquad & 14 & \qquad & \qquad \\ \hline
    15 & \qquad & \qquad & 16 & \qquad & \qquad \\ \hline
    \end{tabularx}
\end{table}
\begin{table}[!htbp]
    \caption{解答题情况}
    \centering
    \begin{tabularx}{0.5\textwidth}{|*{3}{>{\centering\arraybackslash}X|}} \hline
    \end{tabularx}
\end{table}
\end{document}