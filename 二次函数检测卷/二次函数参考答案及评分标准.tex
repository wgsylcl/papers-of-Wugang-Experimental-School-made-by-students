\documentclass[10pt]{article}
\usepackage{mathpaper}
\usepackage{tabularx}

\begin{document}
\showsecret
\papertitle{第二十二章~~~二次函数}
\centerline{\Large 参考答案及评分标准}

\textbf{\selectingintroduction}
\begin{table}[!htb]
    \centering
    \begin{tabularx}{\textwidth}{|*{11}{>{\centering\arraybackslash}X|}} \hline
        题号 & 1 & 2 & 3 & 4 & 5 & 6 & 7 & 8 & 9 & 10 \\ \hline
        选项 & A & B & C & C & B & B & A & B & C & C \\ \hline
    \end{tabularx}
\end{table}
\par \textbf{\complitingintroduction}
\begin{table}[!htb]
    \centering
    \renewcommand\arraystretch{1.5}
    \begin{tabularx}{\textwidth}{*{3}{>{\centering\arraybackslash}X}}
        11.$\ (-1,0)和\left(-\frac{1}{2},0\right)$ & 12.$\ (-3,-2)$ & 13.$\ -10或-2$ \\
        14.$\ 6$ & 15.$\ $\circnum{1} \circnum{2} \circnum{3} \circnum{4} \circnum{5} & 16.$\ -\frac{9}{4} \leq x < -2$  \\
    \end{tabularx}
\end{table}

\setcounter{taskcounter}{16}
\begin{questions}{\answeringintroduction}
    \question 设抛物线解析式为$y=ax^2+bx+c$,依题意,有
    $$\begin{cases}
        2 = 4a - 2b + c \\
        2 = 9a + 3b + c \\
        -4 = 4a + 2b + c
    \end{cases}$$
    解之,得
    $$\begin{cases}
        a = \frac{3}{2} \\
        b = -\frac{3}{2} \\
        c = -7
    \end{cases} \ \ \ (2分)$$
    故抛物线解析式为$y=\frac{3}{2}x^2-\frac{3}{2}x-7$(4分),化简得
    $$y=\frac{3}{2}\left(x-\frac{1}{2}\right)^2-\frac{59}{8}$$
    即原抛物线顶点为$\left(\frac{1}{2},-\frac{59}{8}\right)$.(8分)
    \question %18
    \begin{subquestions}
        \subquestion 由题意,因为直线两两不平行,且任意三线不共点,故有
        $$y=\frac{x(x-1)}{2}=\frac{1}{2}x^2-\frac{1}{2}x \ \ \ (2分)$$
        因为$x$表示点数,故$x > 0$(3分)、$x$是整数(4分).
        \subquestion 我们令$y=0$,可以得到
        $$x^2-x-36=0 \ \ \ (5分)$$
        则$a=1、b=1、c=-36$,故
        $$\Delta = b^2-4ac = 1297 \ \ \ (6分)$$
        由求根公式$x=\frac{-b \pm \sqrt{b^2-4ac}}{2a}$知,$x$不是整数 (7分)\\
        故交点个数不可能为$18$.(8分)
    \end{subquestions}
    \question %19
    \begin{subquestions}
        \subquestion $y=x^2-2x$(2分)
        \subquestion 令$y=0$,有$x^2-2x=0$,解得$x_1=2、x_2=0$,即$A(2,0)$.(3分)\\
        记$AB$与$x$轴的交点为$T$,由$\angle BAO=45^{\circ}$易得$T(0,2)$,于是设$AT:y=kx+b$,则
        $\begin{cases}
            2k+b=0 \\
            b=2
        \end{cases}$,
        解得
        $\begin{cases}
            k=-1 \\
            b=2
        \end{cases}$
        即
        $$AT:y=-x+2 \ \ \ (4分)$$
        设点$E(e,e^2-2e)$,并过$E$作$EC \bot EF$交$AB$于$C$,于是$C(e,-e+2)$ (5分)\\
        又因为$EF$平行于$x$轴,故易得$CE=EF$(6分)\\
        于是
        $$\begin{aligned}
            EF = CE &= y_C - y_E \\
                    &= (-e+2) - (e^2-2e) \\
                    &= -e^2+e+2 \\
                    &= -\left(e-\frac{1}{2}\right)^2+\frac{9}{4}
        \end{aligned}$$
        即$EF$最大为$\frac{9}{4}$.(8分)
        \begin{figure}[!htb]
            \raggedleft
                \begin{tikzpicture}[>=Stealth,scale=0.65]
                    \draw[->] (-2,0) -- (4,0) node[below] {$x$};
                    \draw[->] (0,-2) -- (0,4) node[right] {$y$};
                    \coordinate[label=below left:{$O$}] (O) at (0,0);
                    \draw (-1.12,3.5) parabola bend (1,-1) (3.12,3.5);
                    \coordinate[label=below right:{$A$}] (A) at (2,0);
                    \coordinate[label=below left:{$B$}] (B) at (-1,3);
                    \coordinate[label=below left:{$E$}] (E) at (-0.41,1);
                    \coordinate[label=below left:{$F$}] (F) at (1,1);
                    \coordinate[label=right:{$T$}] (T) at (0,2);
                    \coordinate[label=above:{$C$}] (C) at (-0.41,2.41);
                    \draw (-1.5,3.5) -- (3.5,-1.5);
                    \draw (-1.5,1) -- (3.5,1);
                    \draw[densely dashed] (E) -- (C);
                \end{tikzpicture}
        \end{figure}
    \end{subquestions}
    \question %20
    \begin{subquestions}
        \subquestion 令$y=0$,可得$x^2+ax+2a=0$,由题意,$\left|x_1-x_2\right|=3$,即$(x_1-x_2)^2=9$.(1分)\\
        由韦达定理,可得
        $$\begin{cases}
            x_1+x_2=-a \\
            x_1x_2=2a
        \end{cases}\ \ \ (2分)$$
        故有
        $$\begin{aligned}
            (x_1-x_2)^2 &= 9 \\
            (x_1+x_2)^2 - 4x_1x_2 &= 9 \\
            a^2-8a &= 9 \\
            (a-9)(a+1) &= 0
        \end{aligned}$$
        于是$a-9=0$或$a+1=0$,即$a_1 = 9$(舍去)、$a_2 = -1$. \\
        故$a=-1$.(4分)
        \subquestion 当$b < -\frac{9}{2}$时,$y_{min} = b^2+9b+18$.(5分) \\
        当$-\frac{9}{2} \leq b < \frac{1}{2}$时,$y_{min} = -\frac{9}{4}$.(6分) \\
        当$b \geq \frac{1}{2}$时,$y_{min} = b^2-b-2$.(7分) \\
        (注:三种情况全部正确得8分)
    \end{subquestions}
    \question %21
    \begin{subquestions}
        \subquestion 令$y=0$,得到$-\frac{1}{2}x^2-\frac{3}{2}x+c=0$,即$x^2+3x-2c=0$,则$A=1、B=3、C=-2c$.(2分) \\
        因为抛物线与$x$轴有两个不同的交点,故该方程有两个不等实根,故
        $$\begin{aligned}
            \Delta = B^2 - 4AC &> 0 \\
            3^2 - 4 \cdot (-2c) &> 0 \\
            9 + 8c &> 0 \\
            c &> -\frac{9}{8} \ \ \ (4分)
        \end{aligned}$$
        \subquestion 由题,$y=-\frac{1}{2}x^2-\frac{3}{2}x+c=-\frac{1}{2}\left(x+\frac{3}{2}\right)^2+c+\frac{9}{8}$,则$D\left(-\frac{3}{2},c+\frac{9}{8}\right)$,令$x=0$,得$y=c$,故$C(0,c)$.(5分) \\
        设$CD:y=kx+b$,则
        $\begin{cases}
            c = b \\
            c + \frac{9}{8} = -\frac{3}{2}k + b
        \end{cases}$,
        解得
        $\begin{cases}
            k = -\frac{3}{4} \\
            b = c
        \end{cases}$,故$CD:y=-\frac{3}{4}x+c$,又因为$MN$平行于$CD$且过原点,故$MN:y=-\frac{3}{4}x$.(6分) \\
        联立抛物线与$MN$的方程
        $$\begin{cases}
            y = -\frac{4}{3}x \\
            y = -\frac{1}{2}x^2-\frac{3}{2}x+c
        \end{cases}$$
        则$-\frac{4}{3}x = -\frac{1}{2}x^2-\frac{3}{2}x+c$,化简得到$x^2+\frac{3}{2}x-2c=0$,由题意,$x_M$、$x_N$是这个方程的两根,故由韦达定理得
        $$\begin{cases}
            x_M + x_N = -\frac{3}{2} \\
            x_Mx_N = -2c
        \end{cases} \ \ \ (7分)$$
        于是
        $$\begin{aligned}
            {MN}^2 &= 16{CD}^2 \\
            (x_M-x_N)^2+(y_M-y_N)^2 &= 16(x_C-x_D)^2+16(y_C-y_D)^2 \\
            (x_M-x_N)^2+\left(-\frac{3}{4}x_M+\frac{3}{4}x_N\right)^2 &= 36+\frac{81}{4} \\
            \frac{25}{16}(x_M-x_N)^2 &= \frac{225}{4} \\
            (x_M+x_N)^2-4x_Mx_N &= 36 \\
            \frac{9}{4} + 8c &= 36 \\
            c &= \frac{135}{32} \ \ \ (8分)
        \end{aligned}$$
        \begin{figure}[!htb]
            \raggedleft
            \begin{tikzpicture}[>=Stealth,scale=0.4]
                \draw[->] (0,-3.5) -- (0,6) node[below right] {$y$};
                \draw[->] (-6,0) -- (3,0) node[below left] {$x$};
                \draw (-5.59,-3) parabola bend (-1.5,5.34) (2.59,-3);
                \coordinate[label=below left:{$O$}] (O) at (0,0);
                \coordinate[label=below right:{$A$}] (A) at (-4.77,0);
                \coordinate[label=below left:{$B$}] (B) at (1.77,0);
                \coordinate[label=right:{$C$}] (C) at (0,4.22);
                \coordinate[label=above:{$D$}] (D) at (-1.5,5.34);
                \coordinate[label=left:{$M$}] (M) at (-3.75,2.81);
                \coordinate[label=right:{$N$}] (N) at (2.25,-1.69);
                \draw (C) -- (D);
                \draw (M) -- (N);
            \end{tikzpicture}
        \end{figure}
    \end{subquestions}
    \question %22
    \begin{subquestions}
        \subquestion 记$w_甲、w_乙$分别为销售甲、乙两种草莓所获得的总利润,依题意
            $$w_甲 = -3x-x+\begin{cases}
                x(-x+14) & (x \leq 8). \\
                6x & (x > 8).
            \end{cases}
            = \begin{cases}
                -x^2+10x & (x \leq 8). \\
                2x & (x > 8).
            \end{cases} \ \ \ (2分)$$
            $$w_乙 = 9y-3y-(12+3y) = 3y-12 \ \ \ (3分)$$
            则
            $$w = w_甲+w_乙 =
            \begin{cases}
                -x^2+10x+3y-12 & (x \leq 8). \\
                2x+3y-12 & (x > 8).
            \end{cases}$$
            答:$w = \begin{cases}
                -x^2+10x+3y-12 & (x \leq 8). \\
                2x+3y-12 & (x > 8).
            \end{cases}$(4分)
            \newpage
        \subquestion 依题意,$x+y=20$,即$y=-x+20$,于是
        $$w = \begin{cases}
            -x^2+7x+48 & (x \leq 8). \\
            -x+48 & (x > 8).
        \end{cases} \ \ \ (5分)$$
        则依题有$w = 30$.\\
        当$x \leq 8$时,有
        $$\begin{aligned}
            -x^2+7x+48 &= 30 \\
            x^2-7x-18  &= 0 \\
            (x-9)(x+2) &= 0
        \end{aligned}$$
        于是$x-9=0$或$x+2=0$,即$x_1=9$、$x_2=-2$,这两个答案都应舍去.(6分)\\
        当$x > 8$时,有
        $$\begin{aligned}
            -x+48 &= 30 \\
            x &= 18 \ \ \
        \end{aligned}$$
        答:用于销售甲类的草莓有$18$吨.(7分)
        \subquestion 依题,有$3x+x+3y+12+3y=100$,即$3y=44-2x$. \\
        于是有
        $$w = \begin{cases}
            -x^2+8x+32 & (x \leq 8). \\
            32 & (x > 8).
        \end{cases} \ \ \ (8分)$$
        当$x \leq 8$时,有$w = -x^2+8x+32 = -(x-4)^2 + 48$,因为$4 \leq 8$,所以$w_{min} = w\mid_{x=4} = 48$. \\
        当$x > 8$时,$w$恒为$32$. \\
        因为$48 > 32$,故$w_{min} = w\mid_{x=4} = 48$,此时$y=12$. \\
        答:甲类草莓$4$吨,乙类草莓$12$吨,这样有最大利润为$48$万元.(10分)
    \end{subquestions}
    \question %23
    \begin{subquestions}
        \subquestion 联立抛物线与$AB$的方程
        $$\begin{cases}
            y = -\frac{1}{2}x+3 \\
            y = \frac{1}{2}x^2
        \end{cases}$$
        于是有
        $$\begin{aligned}
            \frac{1}{2}x^2 &= -\frac{1}{2}x+3 \\
            x^2+x-6 &= 0 \\
            (x+3)(x-2) &= 0
        \end{aligned}$$
        故$x+3=0$或$x-2=0$,即$x_1=-3$、$x_2=2$,又因为$A$在$B$的左侧,故$A\left(-3,\frac{9}{2}\right)$、$B(2,2)$.(2分)
        \subquestion 设$P\left(p,\frac{1}{2}p^2\right)$,过$P$作$PT \bot x$轴、交$AB$于$T$,则$T\left(p,-\frac{1}{2}p+3\right)$,于是$$PT = y_T - y_P = \left(-\frac{1}{2}p+3\right) - \frac{1}{2}p^2 = -\frac{1}{2}p^2 - \frac{1}{2}p + 3$$
        则
        $$\begin{aligned}
            S_{\Delta ABP} &= \frac{1}{2}PT(x_B - x_A) \\
                           &= -\frac{5}{4}p^2 - \frac{5}{4}p + \frac{15}{2} \ \ \ (4分)
        \end{aligned}$$
        又因为$S_{\Delta ABP} = 5$,故
        $$\begin{aligned}
            -\frac{5}{4}p^2 - \frac{5}{4}p + \frac{15}{2} &= 5 \\
            p^2+p-2 &= 0 \\
            (p+2)(p-1) &= 0
        \end{aligned}$$
        于是$p+2=0$或$p-1=0$,即$p_1=-2$、$p_2=1$,故$P_1(-2,2)$、$P_2\left(1,\frac{1}{2}\right)$.(6分)
        \begin{figure}[!htb]
            \raggedleft
            \begin{tikzpicture}[>=Stealth,scale=0.5]
                \draw[->] (0,-1) -- (0,7) node[below right] {$y$};
                \draw[->] (-4,0) -- (4,0) node[below left] {$x$};
                \draw (-3.46410,6) parabola bend (0,0) (3.46410,6);
                \coordinate[label=below left:{$O$}] (O) at (0,0);
                \coordinate[label=below left:{$A$}] (A) at (-3,4.5);
                \coordinate[label=below right:{$B$}] (B) at (2,2);
                \coordinate[label=above right:{$T$}] (T) at (-1.33,2.33);
                \coordinate[label=below left:{$P$}] (P) at (-1.33,0.89);
                \draw (A) -- (B) -- (P) -- cycle;
                \draw[densely dashed] (-1.33,-1) -- (-1.33,7);
            \end{tikzpicture}
        \end{figure}
        \subquestion
        $\begin{cases}
            P_1\left(-1,\frac{1}{2}\right) \\
            Q_1(0,6)
        \end{cases}$(7分)
        $\begin{cases}
            P_2\left(5,\frac{25}{2}\right) \\
            Q_2(0,15)
        \end{cases}$(8分)
        $\begin{cases}
            P_3\left(-5,\frac{25}{2}\right) \\
            Q_3(0,10)
        \end{cases}$(9分)\\
        (注:三种情况全部正确得10分)
    \end{subquestions}
    \question %24
    \begin{subquestions}
        \subquestion 因为抛物线对称轴为$x=1$、且与$x$轴交于$A(4,0)$和$B$,故
        $$\begin{cases}
            -\frac{1}{2a} = 1 \\
            16a + 4 + c = 0
        \end{cases}$$
        解得
        $$\begin{cases}
            a = -\frac{1}{2} \\
            c = 4
        \end{cases}$$
        故原抛物线解析式为$y=-\frac{1}{2}x^2+x+4$.(2分)
        \subquestion 如图,过$A$作$x$轴的垂线,交$CD$延长线于$T$.\\
        因为抛物线对称轴为$x=1$,且$A(4,0)$,所以$B(-2,0)$.又令$x=0$,得$y=4$,故$C(0,4)$.(3分)\\
        于是$OA=CA=4$,故$\angle OAC = \angle OCA = 45^{\circ}$.
        又因为$CA$平分$\angle BCD$,故$\angle BCA = \angle TCA$,于是$\Delta BCA $\scalebox{-1}[1]{$\cong$}$ \Delta TCA$,故$TA=BA=6$,即$T(4,6)$.(4分) \\
        设$CT:y=kx+b$,则
        $\begin{cases}
            4 = b \\
            6 = 4k + b
        \end{cases}$,解得
        $\begin{cases}
            k = \frac{1}{2} \\
            b = 4
        \end{cases}$,即$CT:y=\frac{1}{2}x+4$.(5分)\\
        联立抛物线与$CT$的方程,得到
        $$\begin{cases}
            y=\frac{1}{2}x+4 \\
            y=-\frac{1}{2}x^2+x+4
        \end{cases}$$
        即
        $$\begin{aligned}
            -\frac{1}{2}x^2+x+4 &= \frac{1}{2}x+4 \\
            x^2-x &= 0 \\
            x(x-1) &= 0
        \end{aligned}$$
        于是$x=0$或$x-1=0$,即$x_1=0$(舍去)、$x_2=1$,所以$D\left(1,\frac{9}{2}\right)$.(7分)
        \begin{figure}[!htb]
            \raggedleft
            \begin{tikzpicture}[>=Stealth,scale=0.5]
                \draw[->] (-3,0)--(5,0) node[below] {$x$};
                \draw[->] (0,-1.75)--(0,6) node[right] {$y$};
                \coordinate[label=below left:{$O$}] (O) at (0,0);
                \coordinate[label=below left:{$A$}] (A) at (4,0);
                \coordinate[label=below right:{$B$}] (B) at (-2,0);
                \coordinate[label=above left:{$C$}] (C) at (0,4);
                \coordinate[label=above right:{$D$}] (D) at (1,4.5);
                \coordinate[label=right:{$T$}] (T) at (4,6);
                \draw (A)--(C)--(B);
                \draw (C)--(D);
                \draw[densely dashed] (A) -- (T);
                \draw[densely dashed] (D) -- (T);
                \draw (-2.5,-1.625) parabola bend (1,4.5) (4.5,-1.625);
            \end{tikzpicture}
        \end{figure}
        \newpage
        \subquestion 依题,设$P(p,5)$,直线$l:y=kx+b$过$P$,则$5=kp+b$,即$b=5-kp$,故$l:y=kx-kp+5$,即
        $$l:y=k(x-p)+5 \ \ \ (8分)$$
        联立$l$与抛物线的方程
        $$\begin{cases}
            y=k(x-p)+5 \\
            y=-\frac{1}{2}x^2+x+4
        \end{cases}$$
        则
        $$\begin{aligned}
            k(x-p)+5 &= -\frac{1}{2}x^2+x+4 \\
            x^2+(2k-2)x+(2-2kp) &= 0 \ \ \ (9分)
        \end{aligned}$$
        当$l$与抛物线有且仅有一个交点时,此方程两根相同,又因为$A=1$、$B=2k-2$、$C=2-2kp$,故
        $$\begin{aligned}
            \Delta = B^2 - 4AC &= 0 \\
            (2k-2)^2-4(2-2kp) &= 0 \\
            k^2+(2p-2)k-1 &= 0 \ \ \ (10分)
        \end{aligned}$$
        由韦达定理,得
        $$\begin{cases}
            k_1+k_2 = -(2p-2) \\
            k_1k_2 = -1
        \end{cases} \ \ \ (*)$$
        于是可设$PC:y=k_1(x-p)+5$、$PA:y=k_2(x-p)+5$. \\
        令$x=1$,得$N(1,k_1(1-p)+5)$、$M(1,k_2(1-p)+5)$\\
        设$G(1,g)$,则
        $$GM = k_2(1-p)+(5-g)、
        GN = k_1(1-p)+(5-g)、
        {GP}^2 = (x_P-x_G)^2 + (y_P-y_G)^2 = (p-1)^2 + (5-g)^2.(11分)$$
        又因为${GP}^2 = GM \cdot GN$,故
        $$\begin{aligned}
            (p-1)^2+(5-g)^2 &= [k_1(1-p)+(5-g)][k_2(1-p)+(5-g)] \\
            (p-1)^2+(5-g)^2 &= k_1k_2(1-p)^2+(k_1+k_2)(1-p)(5-g)+(5-g)^2 \\
            (p-1)^2 &= k_1k_2(1-p)^2+(k_1+k_2)(1-p)(5-g)
        \end{aligned}$$
        将$(*)$代入,有
        $$\begin{aligned}
            (p-1)^2 &= -(1-p)^2-2(p-1)(1-p)(5-g) \\
            2(p-1)^2(4-g) &= 0
        \end{aligned}$$
        由于当$P$运动时,上式恒成立,故$4-g=0$,即$g=4$.\\
        综上所述,$G(1,4)$.(12分)
        \begin{figure}[!htb]
            \raggedleft
            \begin{tikzpicture}[>=Stealth,scale=0.5]
                \draw[->] (-3,0)--(5,0) node[below] {$x$};
                \draw[->] (0,-1.75)--(0,6) node[right] {$y$};
                \coordinate[label=below left:{$O$}] (O) at (0,0);
                \coordinate[label=below left:{$A$}] (A) at (4,0);
                \coordinate[label=below right:{$B$}] (B) at (-2,0);
                \coordinate[label=above left:{$C$}] (C) at (0,4);
                \coordinate[label=above left:{$M$}] (M) at (1,9.79);
                \coordinate[label=above right:{$P$}] (P) at (2.46,5);
                \coordinate[label=left:{$N$}] (N) at (1,4.56);
                \coordinate[label=below right:{$G$}] (G) at (1,4);
                \draw[densely dashed] (1,-1.7)--(1,10.75);
                \draw[densely dashed] (G)--(P);
                \draw (P)--(-1.67,3.75);
                \draw (M)--(4.23,-0.72);
                \draw (-3,5)--(5,5);
                \draw (-2.5,-1.625) parabola bend (1,4.5) (4.5,-1.625);
            \end{tikzpicture}
        \end{figure}
    \end{subquestions}
\end{questions}
\end{document}