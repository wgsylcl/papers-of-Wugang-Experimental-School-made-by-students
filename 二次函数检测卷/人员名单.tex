\documentclass[11pt]{article}
\usepackage[UTF8]{ctex}
\usepackage{booktabs}
\usepackage{multicol}
\usepackage[a4paper,left=10mm,right=10mm,top=10mm,bottom=15mm]{geometry}
\title{\heiti 第二十二章~~~二次函数}
\date{}
\author{}
\begin{document}
\maketitle
% Table generated by Excel2LaTeX from sheet 'Sheet1'
\begin{table}[htbp]
  \centering
    \begin{tabular}{cccccc}
    \toprule
    题号          & 命题          & 审核+修改       & 试做          & 建议          & 建议难度 \\
    \midrule
    1           & pjz         & lcl         & qht         & 不提供         & 1 \\
    \midrule
    2           & pjz         & lcl         & qht         & 不提供         & 1 \\
    \midrule
    3           & pjz         & lcl         & qht         & 不提供         & 1 \\
    \midrule
    4           & pjz         & lcl         & qht         & 不提供         & 1 \\
    \midrule
    5           & qht         & lcl         & lcl         & 不提供         & 2 \\
    \midrule
    6           & qht         & lcl         & lcl         & 不提供         & 2 \\
    \midrule
    7           & qht         & lcl         & lcl         & 不提供         & 2 \\
    \midrule
    8           & qht         & lcl         & lcl         & 不提供         & 2 \\
    \midrule
    9           & llq         & lcl         & qht         & 不提供         & 2 \\
    \midrule
    10          & fjh         & lcl         & qht         & 抛物线与几何综合    & 3 \\
    \midrule
    11          & qht         & fjh         & lcl         & 不提供         & 1 \\
    \midrule
    12          & qht         & fjh         & lcl         & 不提供         & 2 \\
    \midrule
    13          & pjz         & fjh         & qht         & 不提供         & 2 \\
    \midrule
    14          & pjz         & fjh         & qht         & 不提供         & 2 \\
    \midrule
    15          & pjz         & fjh         & qht         & 二次函数综合      & 2或3 \\
    \midrule
    16          & lcl         & fjh         & qht         & 二次函数与一元二次方程根的关系 & 3 \\
    \midrule
    17          & pjz         & lcl         & qht         & 待定系数求二次函数   & 1 \\
    \midrule
    18          & pjz         & lcl         & qht         & 二次函数最值问题    & 2 \\
    \midrule
    19          & qht         & lcl         & lcl         & 抛物线简单计算     & 2 \\
    \midrule
    20          & qht         & lcl         & lcl         & 抛物线与简单几何    & 2 \\
    \midrule
    21          & llq         & lcl         & qht         & 二次函数与增长、利润问题 & 2 \\
    \midrule
    22          & llq         & lcl         & qht         & 实物抛物线       & 2 \\
    \midrule
    23          & fjh         & lcl         & qht         & 抛物线与几何综合    & 3 \\
    \midrule
    24          & lcl         & lcl         & qht         & 二次函数综合      & 3 \\
    \bottomrule
    \end{tabular}%
\end{table}%
\end{document}