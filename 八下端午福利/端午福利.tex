\documentclass[11pt]{article}
\usepackage[UTF8]{ctex}
\usepackage[a4paper,left=25mm,right=25mm,top=25mm]{geometry}
\usepackage{xeCJKfntef}\usepackage{amsmath, amsfonts,amssymb}   %数学符号
%选择题盒子,1行4选项
\newcommand{\onp} [4] { \\
    \begin{tabular} {*{4}{@{}p{3.5cm}}}
        A.~#1 & B.~#2 & C.~#3 & D.~#4
    \end{tabular}
}
%选择题盒子,1行2选项
\newcommand{\twp} [4] { \\
    \begin{tabular} {*{2}{@{}p{7cm}}}
        A.~#1 & B.~#2
    \end{tabular} \\
    \begin{tabular} {*{2}{@{}p{7cm}}}
        C.~#3 & D.~#4
    \end{tabular}
}
%选择题盒子,1行1选项
\newcommand{\fop} [4] { \\
    A.~#1 \\
    B.~#2 \\
    C.~#3 \\
    D.~#4
}
%图片盒子 参数1为路径 2为大小
\newcommand{\smallpicture}[2]{\includegraphics[scale = #2]{#1}}
\usepackage{graphicx}  %插图包
\usepackage{paralist}  %序号包
%定位器 参数1、2为横纵坐标 3为插入盒子
\newcommand{\pbox} [3] {
    \unitlength=1mm
    \begin{picture} (0, 0)
        \put (#1, #2) {#3}
    \end{picture}
}
\begin{document}
\section*{\centering 端午福利}
\section*{\normalsize 一、选择题(每小题3分)}
\begin{enumerate}\setcounter{enumi}{0}
    \item %1
    \item %2
    \item %3
    \item %4
    \item %5
    \item %6
    \item %7
    \item %8
    \item %9
    \item %10
\end{enumerate}
\section*{\normalsize 二、填空题(每小题3分)}
\begin{enumerate}\setcounter{enumi}{10}
    \item %11
    \item %12
    \item %13
    \item %14
    \item %15
    \item %16
\end{enumerate}
\section*{\normalsize 三、解答题(共8题,每小题应写出文字说明、解答过程或演算步骤)}
\begin{enumerate}\setcounter{enumi}{16}
    \item %17
    \begin{compactenum}[(1)]
        \item %17.1
        \item %17.2
    \end{compactenum}
    \item %18
    \begin{compactenum}[(1)]
        \item %18.1
        \item %18.2
    \end{compactenum}
    \item %19
    \begin{compactenum}[(1)]
        \item %19.1
        \item %19.2
    \end{compactenum}
    \item %20
    \begin{compactenum}[(1)]
        \item %20.1
        \item %20.2
    \end{compactenum}
    \item %21
    \begin{compactenum}[(1)]
        \item %21.1
        \item %21.2
    \end{compactenum}
    \item %22
    \begin{compactenum}[(1)]
        \item %22.1
        \item %22.2
        \item %22.3
    \end{compactenum}
    \item %23
    \begin{compactenum}[(1)]
        \item %23.1
        \item %23.2
        \item %23.3
    \end{compactenum}
    \item %24
    \begin{compactenum}[(1)]
        \item %24.1
        \item %24.2
        \item %24.3
    \end{compactenum}
\end{enumerate}
\end{document}