\documentclass[10pt]{article}
\usepackage[UTF8]{ctex}
\usepackage[a4paper,left=25mm,right=25mm,top=25mm]{geometry}
\usepackage{xeCJKfntef}\usepackage{amsmath, amsfonts,amssymb}   %数学符号
%选择题盒子,1行4选项
\newcommand{\onp} [4] { \\
    \begin{tabular} {*{4}{@{}p{3.5cm}}}
        A.~#1 & B.~#2 & C.~#3 & D.~#4
    \end{tabular}
}
%选择题盒子,1行2选项
\newcommand{\twp} [4] { \\
    \begin{tabular} {*{2}{@{}p{7cm}}}
        A.~#1 & B.~#2
    \end{tabular} \\
    \begin{tabular} {*{2}{@{}p{7cm}}}
        C.~#3 & D.~#4
    \end{tabular}
}
%选择题盒子,1行1选项
\newcommand{\fop} [4] { \\
    A.~#1 \\
    B.~#2 \\
    C.~#3 \\
    D.~#4
}
%图片盒子 参数1为路径 2为大小
\newcommand{\smallpicture}[2]{\includegraphics[scale = #2]{#1}}
\usepackage{graphicx}  %插图包
\usepackage{paralist}  %序号包
%定位器 参数1、2为横纵坐标 3为插入盒子
\newcommand{\pbox} [3] {
    \unitlength=1mm
    \begin{picture} (0, 0)
        \put (#1, #2) {#3}
    \end{picture}
}
\begin{document}
\section*{\centering 端午福利}
\section*{\normalsize 一、选择题(每小题3分)}
\begin{enumerate}\setcounter{enumi}{0}
    \item 分式$\frac{\sqrt{(x-1)^2-9}}{x-4}$在实数范围内有意义,则$x$取值范围为(~~~)。
    \onp{$x\neq 1$且$x\neq 2$}{$x\le-2$或$x>4$}{$x>4$}{$x\le3$且$x\neq4$}
    \item 下列运算正确的是(~~~)。
    \twp{$(\sqrt{5})^2-(\sqrt{4})^2=(\sqrt{3})^2$}{$\sqrt{2}-\frac{\sqrt{2}}{2}=\sqrt{1}$}{$\sqrt{81}-\sqrt{9}=6$}{$\sqrt{8}+\sqrt{2}=4\sqrt{2}$}
    \item 下列各组数中,为勾股数的是(~~~)。
    \onp{$11,60,61$}{$17,11,13$}{$2,3,5$}{$1,\sqrt{3},2$}
    \item 跳远比赛中,甲、乙、丙、丁各跳五次的成绩如图所示,若发挥优秀且稳定者胜出,则应选(~~~)为冠军。\\
    \smallpicture{T4}{1}\\
    \onp{甲}{乙}{丙}{丁}
    \item 已知一次函数$y=-(n^2-2n+1)x-n^2+2n-1$,则下列结论错误的是(~~~)。
    \fop{函数图像过一、二、四象限}{图像恒过点$(-1,0)$}{$y$随$x$增大而减小}{当$n=0$时,图像与坐标轴围成的三角形面积为$\frac{1}{2}$}
    \item 直角三角形$ABC$,$\angle C=90^\circ,\angle ABC=\frac{2}{3}\angle ABD=30^\circ$,则下列选项正确的是(~~~)。
    \twp{$AC=2CD$}{$CD=\frac{2\sqrt{3}-3}{2}AB$}{$S_{\Delta ABD}=\sqrt{3}AD$}{$AC\times CD=BC^2$}\\
    \smallpicture{T6}{0.05}
    \item 如图,直线$l_1:y=k_1x+b_1,l_2:y=k_2x+b_2$交于点$P(3,m)$则当$k_2x+b_2>k_1x+b_1$时,$x$取值范围为(~~~)。
    \onp{$x>3$}{$x\ge m$}{$x<m$}{$x\ge 3$}\\
    \smallpicture{T7}{0.06}\smallpicture{T8}{0.06}
    \item 小李放学后立即回家,她在中途去小卖铺买了点东西,用时$10min$。她步行速度为$50m/min$。小王在$15$分钟后做完清洁,发现小李把数学作业落在学校了,于是以$100m/min$的速度去追小李,给她送作业。其路程与时间的关系的关系如图所示,则(~~~)。
    \fop{$a=4$}{$P(20,5)$}{两人相距$200m$时,$t=4$或$18$}{若小李家距学校$1000m$,则小李$20$分钟到家}
    \item 已知$6$个互不相同的数$s_1,s_2,s_3...s_6$,其中$s_1$最小,$s_6$最大,记数据甲$=\{s_1,s_2,s_3...s_6\}$,数据乙$=\{s_2,s_3,s_4,s_5\}$,则甲和乙的(~~~)一定相等。
    \onp{方差}{平均数}{中位数}{极差}
    \item 记$[x]$为不超过$x$的最大整数,若方程$kx=[x]$有且仅有两个解,则(~~~)。
    \twp{$\frac{1}{2}<k\le \frac{2}{3}$}{$\frac{2}{3}<k\le \frac{3}{4}$}{$\frac{1}{2}<k\le \frac{2}{3}$或$k\ge 2$}{$\frac{2}{3}<k\le \frac{3}{4}$或$\frac{3}{2}\le k<2$}
\end{enumerate}
\section*{\normalsize 二、填空题(每小题3分)}
\begin{enumerate}\setcounter{enumi}{10}
    \item 计算:$\sqrt{\frac{3^2}{2^3}}=\_\_\_\_\_$。
    \item 将直线$l:y=(k-1)x+b+3$向上平移两个单位,向右平移$4$个单位后得到的解析式为\_\_\_\_\_\_\_\_\_\_\_\_\_\_\_。
    \item 数学考试中,小王一学期的分数如下所示:$96,108,104,107,99,104$ 则方差为\_\_\_\_\_\_\_。
    \item 小方立定跳远的成绩如下所示:$1.2,1.3,1.4,1.4,1.5,1.6,1.3,1.4,1.5$ 则其平均数为\_\_\_\_\_\_\_,中位数为\_\_\_\_\_\_\_。
    \item 如图,$l_1:y=3x+3$交$x,y$轴于点$A,C$,$CB$垂直于$AC$且$CB=AC$,将线段$AC$沿$l_2$向右平移至$BD$交$x$轴于$Q$,连$AD$。$P$为$l_2$上一动点,连接$AP$,$M$为$AP$中点。连接$CM,BM$则下列选项正确的有\_\_\_\_\_\_\_。\\
    a.四边形$ACBD$是平行四边形。\\
    b.当$P$为$BC$中点时,$\angle PAQ=45^\circ$。\\
    c.$\angle PAC=45^\circ$,直线$AP$的比例系数为$\frac{1}{2}$或$-3$。\\
    d.当$CM+MB$最小时,$\Delta CMB$为等腰直角三角形。
    \item 如图,平行四边形$ABCD$中,$\angle ABC=60^\circ$,$M$、$E$为平行四边形$ABCD$内两点,连$AM$、$BM$、$ME$、$DE$、$CE$,若$\angle DEC=90^\circ,AB=8,AD=10$,则$AM+BM+EM$最小值为\_\_\_\_\_\_\_。\\
    \smallpicture{T15}{0.15}\smallpicture{T16}{0.5}
\end{enumerate}
\section*{\normalsize 三、解答题(共8题,每小题应写出文字说明、解答过程或演算步骤)}
\begin{enumerate}\setcounter{enumi}{16}
    \item 计算(第二小问解方程)。
    \begin{compactenum}[(1)]
        \item $\frac{\sqrt{54}+3\sqrt{2}}{\sqrt{6}-\sqrt{2}}$
        \item $\frac{\sqrt{3x^2+\sqrt{48}x+4}}{x-3}=\frac{\sqrt{96}+\sqrt{32}}{-\sqrt{8}}~~~(x\ge -\frac{2}{3}\sqrt{3})$
    \end{compactenum}
    \\ \hspace*{\fill} \\\\ \hspace*{\fill} \\\\ \hspace*{\fill}
    \item 正方形$ABCD$中,边$CD$上有一点$T$,$AT$垂直于$TQ$且$AT=TQ$,延长$DC$至$N$。
    \begin{compactenum}[(1)]
        \item 如图,证明:$Q$在$\angle BCN$角平分线上。\pbox{20}{-42}{\smallpicture{T18}{0.4}}
        \item 当$CD=38$,$TD=18$时,求$DQ$长。
    \end{compactenum}
    \\ \hspace*{\fill} \\\\ \hspace*{\fill} \\\\ \hspace*{\fill} \\\\ \hspace*{\fill} \\\\ \hspace*{\fill}
    \item 在平面直角坐标系内,有$A(0,a)$、$B(b,0)$、$C(-2,0)$三点。$D$为$AB$上一点,连$CD$交$y$轴为$E$。已知$a=\sqrt{3-b}+\sqrt{b-3}+4$,$D$的横坐标为$t$。
    \begin{compactenum}[(1)]
        \item 写出$E$点的坐标(用含$t$的式子表示)。\pbox{27}{-30}{\smallpicture{T19}{0.5}}
        \item 当$S_{\Delta ECO}=S_{\Delta AED}$时,求$D$点坐标。
    \end{compactenum}
    \\ \hspace*{\fill} \\\\ \hspace*{\fill} \\\\ \hspace*{\fill} \\\\ \hspace*{\fill} \\\\ \hspace*{\fill}
    \item 某校期末考试结果如下表:\\
    \smallpicture{T20-1}{0.45}\smallpicture{T20-2}{0.45}\\
    \smallpicture{T20-3}{0.45}\\
    \begin{compactenum}[(1)]
        \item 补全表格并计算饼图中$D$的圆心角$\alpha$的度数。
        \item 若$C$,$D$等级为不合格,请计算该校期末考试的合格率。
        \item 本校某学生成绩如下表:\\
        \\\smallpicture{T20-4}{0.0875}\\
        若总成绩按平时、期中、期末$12:13:14$计算(平时成绩取平均分),则$x$至少为多少才能保证总成绩不低于$95$分,求出此时$x$的值(取整数)。
    \end{compactenum}
    \\ \hspace*{\fill} \\\\ \hspace*{\fill} \\\\ \hspace*{\fill} \\\\ \hspace*{\fill}
    \item 如图为$8×8$的网格图,$O,A,B,N$为格点,$AB$交格线于$C,E$,平面直角坐标系$xOy$,$A(0,4)$,$B(1,0)$,$AB$交直线$y=1$于$E$,交直线$y=2$于$C$,$N(4,5)$。
    \begin{compactenum}[(1)]
        \item 直接写出$AE$与$AN$的关系。
        \item 作平行四边形$ACND$,过$B$作直线$l$平分平行四边形$ACND$的面积,写出$l$的解析式。
        \item 在$CN$上找一点$Q$,使$AQ=CQ$。
        \item 画线段$AB$绕$B$顺时针旋转$90^\circ$后的线段$TB$并在$AN$上取一点$F$使$EF=BE+NF$。
        \item 将$AB$绕$A$逆时针旋转$120^\circ$得到$AM$,直接写出$M$的坐标(不必画出来)。\\
        \smallpicture{T21}{0.55}
    \end{compactenum}
    \item ``一方有难,八方支援''。疫情期间,某养殖场鸡蛋滞销,批发商分别以$m$元$/kg$、$n$元$/kg$收购$A$、$B$两种鸡蛋。已知购进$750 kg~A$和$650 kg~B$需要$20700$元;购进$650kg~A$和$750kg~B$需要$21300$元。为感谢批发商,养殖场将$A$价格减少$k$元$/kg$($0<k<3$)。批发商分别将$A$、$B$两种鸡蛋按原购进价格提升$40\%$和$30\%$出售。
    \begin{compactenum}[(1)]
        \item 批发商打算购进$A$、$B$共$1t$,其中$A$不少于$50kg$,不多于$B$的两倍($A$、$B$均为整数)。问如何进货才能使总利润$w$最大,并写出$w$(用含$k$的式子表示)。
        \item 正值希望工程,批发商决定每售出$1kgB$鸡蛋便捐出$a$元($0<a<5.2$),当最大利润为$4733.2$元时,求$a$的值(在($1$)的条件下且不考虑$k$值)。
        \item 为了保鲜,批发商将$2000kg$冰袋运往$C$、$D$两市仓库,由$E$运向$C$的运费为$1$元$/kg$,运向$D$为$4$元$/kg$;由F运向C的运费为$2$元$/kg$,运向$D$为$2.5$元$/kg$。现欲往$C$运送冰袋$xkg$,补齐下表并写出总运费和$x$的关系。\\
        \smallpicture{T22}{0.75}
    \end{compactenum}
    \\ \hspace*{\fill} \\\\ \hspace*{\fill} \\\\ \hspace*{\fill} \\\\ \hspace*{\fill} \\\\ \hspace*{\fill}
    \item %23
    \begin{compactenum}[(1)]
        \item 菱形$ABDC$中,$\angle D=60^\circ$,将$AC$逆时针旋转得到$OC$,连$BO$,$AO$。
        \begin{compactenum}[(i)]
            \item 直接写出$\angle BOA=\_\_\_\_\_\_\_\_\_\_\_\_\_\_\_\_\_\_$。
            \item 如图$1$,$AO=2\sqrt{3}$,$BO=2$,求$CD$。
        \end{compactenum}
        \item 如图$2$,在$\Delta ABC$中,$\angle ABC=80^\circ$,$CD$平行于$AE$,连$AD$,$\angle BAD = 2\angle CAD = 20^\circ$,作$DE$平行于$BC$交$AB$延长线于$E$,证明:四边形$BCDE$是菱形。\par
        {\centering{\smallpicture{T23-1}{0.5}\smallpicture{T23-2}{0.5}}}
    \end{compactenum}
    \\ \hspace*{\fill} \\\\ \hspace*{\fill} \\\\ \hspace*{\fill} \\\\ \hspace*{\fill} \\\\ \hspace*{\fill} \newpage
    \item 已知在平面直角坐标系中,直线$y=-\frac{1}{3}x+2$交$x$轴于$A$,$y$轴于$B$。
    \begin{compactenum}[(1)]
        \item 求$S_{\Delta AOB}$。
        \item 如图,将$AB$绕点$A$顺时针转$135^\circ$,得到直线$AC$交$y$轴于$C$,求$C$点坐标。
        \item 将$CA$向上平移$n$个单位,交直线$y=x+4n$于$M$,$N$为$AB$上一点,直接写出所有的组合$M$、$N$,使得$M$、$N$、$A$、$C$构成的四边形是平行四边形。\par
        {\centering \smallpicture{T24}{0.075}\smallpicture{T24}{0.075}}
    \end{compactenum}
\end{enumerate}
\end{document}