\documentclass[11pt]{article}
\usepackage{mathpaper}
\title{自制小题}
\date{2023年8月3日}
\author{lcl}

\begin{document}
\maketitle
\section*{题目}
甲、乙、丙三人玩传花游戏,游戏的规则是:在每一回合的游戏中,有一人拿到花,并随机传给两人中的另一个,接到花的人成为下一回合拿花的人。为了保证游戏公平,决定在每一回合中,使用掷硬币的方式决定将花传给谁,即每个未拿到花的人都有一半的概率在下一回合拿到花。现在假设游戏从甲开始,即第一回合甲拿到花,则第$i$轮甲拿到花的概率是多少?(用$i$表示,$i$是一个正整数)
\section*{解答}
解:\\
我们设$a_i,b_i,c_i$分别为$P($第$i$回合甲拿到花$)$、$P($第$i$回合乙拿到花$)$和$P($第$i$回合丙拿到花$)$。\\
由于每一回合游戏都必定有人拿到花,因此$$a_i+b_i+c_i=1$$
同理,有$$a_{i-1}+b_{i-1}+c_{i-1}=1$$
又因为甲拿到花的条件是:上一回合乙拿到花并传给甲或上一回合丙拿到花并传给甲,因此有$$a_i=\frac{1}{2}b_{i-1}+\frac{1}{2}c_{i-1}$$
即$$a_i=\frac{1}{2}-\frac{1}{2}a_{i-1}$$
上式化为$$a_i-\frac{1}{3}=-\frac{1}{2}(a_{i-1}-\frac{1}{3})$$
因此,数列$\{a_i-\frac{1}{3}\}$是公比为$-\frac{1}{2}$的等比数列,且首项为$\frac{2}{3}$,于是有
$$a_i-\frac{1}{3}=\frac{2}{3}(-\frac{1}{2})^{i-1}$$
即$$a_i=\frac{2}{3}(-\frac{1}{2})^{i-1}+\frac{1}{3}$$
综上所述,第$i$回合甲拿到花的概率是$\frac{2}{3}(-\frac{1}{2})^{i-1}+\frac{1}{3}$。
\end{document}