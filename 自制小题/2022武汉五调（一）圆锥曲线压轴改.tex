\documentclass[10pt]{article}
\usepackage{mathpaper}

\begin{document}
\setcounter{taskcounter}{23}
\begin{questions}{}
\question 已知在平面直角坐标系$xOy$中,动点$P$到点$F(0,1)$的距离恒等于其到直线$y=-1$的距离,记$P$的轨迹为$E$.
\begin{subquestions}
    \subquestion 求$E$的解析式.
    \subquestion 已知平面内一直线$l_1$过点$M(-1,-1)$,并与$E$交于$A$、$B$两点,过$A$作比例系数为$-\frac{1}{2}$的直线$l_2$,交$E$于另一点$C$.
    \begin{subsubquestions}
        \subsubquestion 求证:直线$BC$过定点.
        \subsubquestion 记\circnum{1}中的定点为$H$,若$S_{\triangle ABH}$的面积不大于$5$,求直线$l_1$比例系数的取值范围.
    \end{subsubquestions}
\end{subquestions}
\end{questions}

\setcounter{taskcounter}{23}
\begin{questions}{【解析】}
    \question 解:
    \begin{subquestions}
        \subquestion 设$P(x_0,y_0)$,依题意,$x_0^2+(y_0-1)^2=(y_0+1)^2$,化简后得到$y_0=\frac{1}{4}x_0^2$,故$E:y=\frac{1}{4}x^2$.
        \subquestion \begin{subsubquestions}
            \subsubquestion 设$A(4a,4a^2)$、$B(4b,4b^2)$、$C(4c,4c^2)$,$AB:y=k_1x+b_1$,则$\begin{cases}
                4a^2 = 4ak_1+b_1 \\
                4b^2 = 4bk_1+b_1
            \end{cases}$,解得$\begin{cases}
                k_1 = a+b \\
                b_1 = -4ab
            \end{cases}$,故$AB:y=(a+b)x-4ab$. \par
            同理,$BC:y=(b+c)x-4bc$、$CA:y=(c+a)x-4ca$. \par
            $\because AB$过$M(-1,-1)$,$\therefore -1=-a-b-4ab \Rightarrow 4ab+a+b=1$. \par
            $\because k_{AC}=-\frac{1}{2}$,$\therefore c+a=-\frac{1}{2} \Rightarrow a=-c-\frac{1}{2}$,代入上式得到$4b(-c-\frac{1}{2})-c-\frac{1}{2}+b=1 \Rightarrow -(b+c)-4bc=\frac{3}{2}$. \par
            $\therefore BC$过$(-1,\frac{3}{2})$.
            \subsubquestion 令$k=a+b$,则$k=k_{AB}$. \par
            依题意,$H(-1,\frac{3}{2})$,则$MH \bot x\text{轴}$,并且$MH=\frac{5}{2}$,则$S_{\triangle ABH}=\frac{1}{2}MH|x_B-x_A|=\frac{5}{4}|4b-4a|=5|a-b|$. \par
            $\because S_{\triangle ABH} \leq 5$,$\therefore |a-b| \leq 1 \Rightarrow (a-b)^2 \leq 1 \Rightarrow (a+b)^2-4ab \leq 1 \Rightarrow (a+b)^2+a+b-1 \leq 1 \Rightarrow k^2+k-2 \leq 0 \Rightarrow -2 \leq k \leq 1$. \par
            $\because AB:y=(a+b)x-4ab=kx+k-1$与抛物线有交点,$\therefore x^2-4kx-4k+4=0$有实数解,于是$16k^2-4(-4k+4) \geq 0 \Rightarrow k^2+k-1 \geq 0$,解得$k \geq \frac{\sqrt{5}-1}{2}$或$k \leq -\frac{\sqrt{5}+1}{2}$. \par
            综上,$\frac{\sqrt{5}-1}{2} \leq k \leq 1$或$-2 \leq k \leq -\frac{\sqrt{5}+1}{2}$.
        \end{subsubquestions}
    \end{subquestions}
\end{questions}
\end{document}