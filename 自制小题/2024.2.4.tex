\documentclass[11pt]{article}
\usepackage{mathpaper}
\title{培训班小题}
\date{2024年2月4日}
\author{lcl}

\begin{document}
\maketitle
\section*{题目}
已知$m$、$n$是两个正整数,且$n \leq 100$,求证:分数$\frac{m}{n}$在化成十进制小数时,其小数点后不可能有连续三位是$1$、$6$、$7$.
\section*{解答}
证明:\par
用反证法,假设$\frac{m}{n}$的十进制小数中小数点后第$(k+1)$、$(k+2)$、$(k+3)$位($k$是正整数)分别为$1$、$6$、$7$,那么可设$\frac{m}{n}=\overline{A.a_1a_2 \dots a_{k-1}a_k167a_{k+4}a_{k+5} \dots}$,其中$A$是正整数、$a_i$($i$为正整数)是小于$10$的自然数,于是有$10^k \cdot \frac{m}{n}=\overline{Aa_1a_2 \dots a_{k-1}a_k}+\overline{0.167a_{k+4}a_{k+5}\dots}$,这样$\frac{10^k \cdot m - n \cdot \overline{Aa_1a_2 \dots a_{k-1}a_k}}{n} = \overline{0.167a_{k+4}a_{k+5}\dots}$ \par
记$b=10^k \cdot m - n \cdot \overline{Aa_1a_2 \dots a_{k-1}a_k}$,显然$b$是一个正整数,那么$\frac{b}{n}=\overline{0.167a_{k+4}a_{k+5}\dots}$,于是有$0.167 < \frac{b}{n} < 0.168$,故$0.167n < b < 0.168n \Rightarrow 1002n < 6000b < 1008n \Rightarrow 2n < 1000(6b-n) <8n$ \par
然而$n$是不大于$100$的正整数,因此显然上式不可能成立,于是假设不成立,因此分数$\frac{m}{n}$在化成十进制小数时,其小数点后不可能有连续三位是$1$、$6$、$7$,证毕.
\end{document}