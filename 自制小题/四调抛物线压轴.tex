\documentclass[10pt]{article}
\usepackage{mathpaper}

\begin{document}
\textbf{【2023 武汉四调】}
\begin{figure}[!htb]
    \centering
    \smallpicture{whsdpwxyz.png}{0.6}
\end{figure}\par

\begin{questions}{}
    \question \textbf{【第3题解答】} \\
    由(1)知$C_1:y=x^2-2x-3=(x-1)^2-4$,$\because C_2$由$C_1$平移得到,且顶点为原点,$\therefore C_2:y=x^2$.\\
    设$M(m,0)$、$PQ:y=kx+b$,则有$0=km+b \Rightarrow b=-km \Rightarrow PQ:y=kx-km$,故$P(p,kp-km)$、$Q(q,kq-km)$ \\
    联立$C_2$与$PQ$的解析式,有
    $\begin{cases}
        y=kx-km \\
        y=x^2
    \end{cases}$,整理得$x^2-kx+km=0$.\\
    $\because x_P$、$x_Q$是这个方程的两根,$\therefore$由韦达定理,有
    $\begin{cases}
        p+q=k \\
        pq=km
    \end{cases} (*)$\\
    (接下来的地方正常情况下应使用相似,此处因为还没学改为勾股定理)\\
    $\because \angle PNQ=90^{\circ} \ \therefore PQ^2=PN^2+QN^2$,即
    $$\begin{aligned}
        (x_P-x_Q)^2+(y_P-y_Q)^2 &= (x_P-x_N)^2+(y_P-y_N)^2+(x_Q-x_N)^2+(y_Q-y_N)^2 \\
        x_P^2-2x_Px_Q+x_Q^2+y_P^2-2y_Py_Q+y_Q^2 &= x_P^2-2x_Px_N+x_N^2+y_P^2-2y_Py_N+y_N^2+x_Q^2-2x_Qx_N+x_N^2+y_Q^2-2y_Qy_N+y_N^2 \\
        x_Px_Q+y_Py_Q &= x_Px_N+y_Py_N+x_Qx_N+y_Qy_N-x_N^2-y_N^2
    \end{aligned}$$
    又$\because N(0,3)$,$\therefore$将$P$、$Q$、$N$三点坐标代入,有
    $$\begin{aligned}
        pq+(kp-km)(kq-km) &= 3(kp-km)+3(kq-km)-9 \\
        pq+k^2(p-m)(q-m) &= 3k(p-m)+3k(q-m)-9 \\
        pq+k^2[pq-m(p+q)+m^2] &= 3k(p+q-2m)-9
    \end{aligned}$$
    将$(*)$代入,有
    $$\begin{aligned}
        km+k^2(km-mk+m^2) &= 3k(k-2m)-9 \\
        (m^2-3)k^2+7mk+9 &= 0
    \end{aligned}$$
    接下来分情况解这个方程:
    \begin{subsubquestions}
        \subsubquestion 当$m^2-3=0$时,有$m=\pm \sqrt{3}$,$\because m>0$,$\therefore m=\sqrt{3}$,
        此时原方程化为$7\sqrt{3}k+9=0$,解得$k=-\frac{3}{7}\sqrt{3}$,符合要求.\\
        \subsubquestion 当$m^2-3 \neq 0$时,由上知$m \neq \sqrt{3}$,此时$A=m^2-3$、$B=7m$、$C=9$ \\
        $\because$直线$PQ$唯一,$\therefore$原方程两根相同,即$\Delta=0$,然而$\Delta=B^2-4AC=13m^2+108>0$,故此情况不成立.\\
    \end{subsubquestions}
    综上所述,$m=\sqrt{3}$.
\end{questions}

\end{document}