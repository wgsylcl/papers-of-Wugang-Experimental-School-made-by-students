\documentclass[10pt]{article}
\usepackage{mathpaper}

\title{魔术揭秘}
\author{lcl}
\date{2024大年初一}

\begin{document}
\maketitle
\section*{内容}
\begin{enumerate}
    \item 拿出$4$张扑克牌,各撕成两半组成两组(每张扑克牌撕出的两张都在不同的组里),并把两组牌叠在一起.
    \item 按照自己名字字数$n$,将手中的牌洗$n$次,每次将最上面一张牌放到\textbf{最下面}.
    \item 将前$3$张牌插入牌组\textbf{中间},完成后将新的牌组中,最上面的扑克牌拿出,并收好.
    \item 根据自己的地域,将牌组前$1$、$2$或$3$张牌插入牌组\textbf{中间}.
    \item 根据自己的性别,男生弃牌组最上方一张牌,女生弃两张.
    \item ``见证奇迹的时刻!'',将手中的牌洗$7$次,每次将最上面一张牌放到\textbf{最下面}.
    \item ``好运留下来,烦恼丢出去!'',将牌组中最上面一张牌放到最下面,然后丢新牌组中最上面的牌,往复循环,直至手中只剩下一张牌.
    \item 手上剩下的半张牌正好和操作3中拿走的半张对应.
\end{enumerate}

\section*{原理}
第一步操作中,只是撕开了扑克牌,没有改变牌组顺序,我们记撕开后的两组扑克牌按顺序为$ABCD$和$abcd$,大小写对应,那么叠起来后为$ABCDabcd$. \par
第二步操作中,轮换的将最上面的牌放到最下面并不会改变牌的结构,具体的,你可以将牌组首尾连接,如下图:\par
\begin{figure}[!htb]
    \centering
    \begin{tikzpicture}[>=Stealth,scale=1.5]
        \tikzset{
            box/.style ={
            circle,
            % minimum width =50pt, %最小宽度
            % minimum height =20pt, %最小高度
            % inner sep=5pt, %文字和边框的距离
            draw=blue %边框颜色
        }}
        \node[box] (A) at (-1,1) {$A$};
        \node[box] (B) at (0,1) {$B$};
        \node[box] (C) at (1,1) {$C$};
        \node[box] (D) at (1,0) {$D$};
        \node[box] (a) at (1,-1) {$a$};
        \node[box] (b) at (0,-1) {$b$};
        \node[box] (c) at (-1,-1) {$c$};
        \node[box] (d) at (-1,0) {$d$};
        \draw (A) -- (B) -- (C) -- (D) -- (a) -- (b) -- (c) -- (d) -- (A);
    \end{tikzpicture}
    \caption*{(扑克牌按顺序组成的环)}
\end{figure}
这样,每次洗牌实际上只是在更改分开这个环的位置,操作结束后,牌组依然保持$ABCDabcd$的结构. \par
第三次操作中,$ABC$被插到中间,$D$被拿走,\textbf{由于$ABC$插到了中间,因此这并不影响最底下一张牌是$d$,这就是关键}. \par
同样的,第四、五次操作后,都不会影响最后一张牌是$d$,此时男、女卡牌的结构分别为$12345d$和$1234d$.\par
第七次操作后,男、女卡牌的结构分别为$2345d1$和$34d12$.\par
第八次操作,本质上是约瑟夫问题,男、女生的操作如下:\par
\begin{itemize}
    \item 男生:$2345d1 \longrightarrow 345d12 \longrightarrow 45d12 \longrightarrow 5d124 \longrightarrow d124 \longrightarrow 124d \longrightarrow 24d \longrightarrow 4d2 \longrightarrow d2 \longrightarrow 2d \longrightarrow d$.
    \item 女生:$34d12 \longrightarrow 4d123 \longrightarrow d123 \longrightarrow 123d \longrightarrow 23d \longrightarrow 3d2 \longrightarrow d2 \longrightarrow 2d \longrightarrow d$.
\end{itemize} \par
因此,无论名字多长、无论南方北方、无论男生女生,最后都剩下$d$,和第三次操作留下的$D$对应.
\end{document}