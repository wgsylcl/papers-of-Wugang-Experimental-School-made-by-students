\documentclass[11pt]{article}
\usepackage{mathpaper}

\title{自制小题}
\date{2024年1月27日}
\author{lcl}

\begin{document}
\maketitle
【2024九省联考T11改】\par
求所有的$f(x)$,使得:
\circnum{1} $f(\frac{1}{2}) \neq 0$;\circnum{2} 对于任意实数$x$、$y$,恒有$f(x+y)+f(x)f(y)=4xy$.\par
【解析】\par
注意到\circnum{2}中等式左边仅含有关函数$f$的项,右边为$4xy$,而\circnum{1}给出条件$f(\frac{1}{2}) \neq 0$,故考虑令$x=\frac{1}{2}$、$y=0$,有$f(\frac{1}{2})+f(\frac{1}{2})f(0)=0$,故$f(0)=-1$.\\

由于已经知道了有关$f(0)$和$f(\frac{1}{2})$的信息,故可考虑令$x=\frac{1}{2}$、$y=-\frac{1}{2}$,这样有$f(0)+f(\frac{1}{2})f(-\frac{1}{2})=-1$,于是$f(-\frac{1}{2})=0$.\par

进一步,注意到\circnum{2}中等式左边有乘积项$f(x)f(y)$,而$f(-\frac{1}{2})=0$,故考虑令$y=-\frac{1}{2}$,有$f(x-\frac{1}{2})+f(x)f(-\frac{1}{2})=-2x \Rightarrow f(x-\frac{1}{2})=-2(x-\frac{1}{2})-1$,用$x$替换掉$x-\frac{1}{2}$,就得到$f(x)=-2x-1$.\par

综上所述,$f(x)=-2x-1$.
\end{document}