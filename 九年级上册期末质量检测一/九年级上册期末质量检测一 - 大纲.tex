\documentclass[10pt]{article}
\usepackage{mathpicture,basicmathpaper}
\usepackage{booktabs,tabularx}

\begin{document}
\papertitle{九年级上册期末质量检测一}
\centerline{\Large 大纲}
\begin{table}[!htb]
    \centering
    \begin{tabularx}{0.8\textwidth}{p{0.1\textwidth}<{\centering}p{0.4\textwidth}<{\centering}p{0.1\textwidth}<{\centering}p{0.1\textwidth}<{\centering}} \toprule
        题号 & 考点 & 难度系数 & 负责人 \\ \hline
        1 & 轴对称和中心对称 & 1 & fjh \\ \hline
        2 & 韦达定理基础 & 1 & lcl \\ \hline
        3 & 旋转基础计算 & 0.9 & fjh \\ \hline
        4 & 随机事件 & 0.9 & lcl \\ \hline
        5 & 抛物线的对称轴 & 0.8 & lcl \\ \hline
        6 & 切线长定理 & 0.8 & fjh \\ \hline
        7 & 二次函数的应用 & 0.8 & lcl \\ \hline
        8 & 圆的中档题 & 0.7 & fjh \\ \hline
        9 & 一元二次方程化简求值 & 0.7 & lcl \\ \hline
        10 & 几何选择难题 & 0.4 & fjh \\ \hline
        11 & 中心对称 & 1 & fjh \\ \hline
        12 & 一元二次方程应用 & 1 & lcl \\ \hline
        13 & 内切圆与外接圆 & 0.9 & fjh \\ \hline
        14 & 二次函数最值 & 0.8 & lcl \\ \hline
        15 & 二次函数多结论 & 0.7 & lcl \\ \hline
        16 & 概率巧题 & 0.3 & lcl \\ \hline
        17 & 二次函数基础计算 & 1 & lcl \\ \hline
        18 & 一元二次方程基础应用 & 0.9 & lcl \\ \hline
        19 & 概率应用 & 0.9 & lcl \\ \hline
        20 & 圆的中档题 & 0.7 & fjh \\ \hline
        21 & 无刻度直尺作图 & 0.7 & fjh \\ \hline
        22 & 二次函数应用(实物抛物线)& 0.75 & lcl \\ \hline
        23 & 旋转综合 & 0.55 & fjh \\ \hline
        24 & 抛物线解析几何 & 0.5 & lcl \\ \bottomrule
    \end{tabularx}
\end{table}
\end{document}