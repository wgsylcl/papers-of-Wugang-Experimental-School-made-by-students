\documentclass[10pt]{article}
\usepackage{mathpaper}

\begin{document}
\papertitle{月考一(第二十一、二十二章)}
\paperinformation{时间:2小时 \ \ \ \ 满分:120分}
\begin{questions}{\selectingintroduction}
    \question %1
    \question %2
    \question %3
    \question %4
    \question %5
    \question %6
    \question %7
    \question %8
    \question %9
    \question %10
\end{questions}
\begin{questions}{\complitingintroduction}
    \question %11
    \question %12
    \question %13
    \question %14
    \question %15
    \question 已知在多项式$x-y-z-m-n$中,有$x>y>z>m>n$,我们称{\kaishu 绝对操作}为:对原多项式任意两个相邻的字母间添加绝对值符号,得到一个新多项式,并对新多项式进行去绝对值运算。\par
    下面是两个{\kaishu 绝对操作}的示例:
    \begin{subsubquestions}
        \subsubquestion $x-y-z-m-n \rightarrow x-y-|z-m|-n = x-y-z+m-n$;
        \subsubquestion $x-y-z-m-n \rightarrow x-|y-z|-|m-n| = x-y+z-m+n$;
    \end{subsubquestions}
    则有如下有关{\kaishu 绝对操作}的说法:
    \begin{subsubquestions}\setcounter{subsubtaskcounter}{0}
        \subsubquestion 存在{\kaishu 绝对操作},使得运算结果与原式相等。
        \subsubquestion 不存在{\kaishu 绝对操作},使得运算结果与原式和为$0$。
        \subsubquestion 所有的{\kaishu 绝对操作}共有$7$种不同的运算结果。
    \end{subsubquestions}
    其中正确的是\complitingline\complitingline。
\end{questions}
\begin{questions}{\answeringintroduction}
    \question %17
    \begin{subquestions}
        \subquestion %17.1
        \subquestion %17.2
    \end{subquestions}
    \question %18
    \begin{subquestions}
        \subquestion %18.1
        \subquestion %18.2
    \end{subquestions}
    \question %19
    \begin{subquestions}
        \subquestion %19.1
        \subquestion %19.2
    \end{subquestions}
    \question
    \begin{subquestions}
        \subquestion %20.1
        \subquestion %20.2
    \end{subquestions}
    \question %21
    \begin{subquestions}
        \subquestion %21.1
        \subquestion %21.2
    \end{subquestions}
    \question %22
    \begin{subquestions}
        \subquestion %22.1
        \subquestion %22.2
        \subquestion %22.3
    \end{subquestions}
    \question %23
    \begin{subquestions}
        \subquestion %23.1
        \subquestion %23.2
        \subquestion %23.3
    \end{subquestions}
    \question %24
    \begin{subquestions}
        \subquestion %24.1
        \subquestion %24.2
        \subquestion %24.3
    \end{subquestions}
\end{questions}
\end{document}