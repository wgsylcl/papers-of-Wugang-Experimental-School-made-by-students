\documentclass[10pt]{article}
\usepackage{mathpaper}

\begin{document}
\showsecret
\anspapertitle{第***章~~~章节名}
\informationline \par
\noindent  \ \textbf{\selectingintroduction}
\begin{table}[!htb]
    \centering
    \begin{tabularx}{\textwidth}{|*{11}{>{\centering\arraybackslash}X|}} \hline
        题号 & 1 & 2 & 3 & 4 & 5 & 6 & 7 & 8 & 9 & 10 \\ \hline
        选项 & \quad & \quad & \quad & \quad & \quad & \quad & \quad & \quad & \quad & \quad \\ \hline
    \end{tabularx}
\end{table}
\par \noindent  \ \textbf{\complitingintroduction}
\begin{table}[!htb]
    \centering
    \renewcommand\arraystretch{1.5}
    \begin{tabularx}{\textwidth}{*{3}{>{\centering\arraybackslash}X}}
        11.\complitingline\complitingline\complitingline & 12.\complitingline\complitingline\complitingline & 13.\complitingline\complitingline\complitingline \\
        14.\complitingline\complitingline\complitingline & 15.\complitingline\complitingline\complitingline & 16.\complitingline\complitingline\complitingline  \\
    \end{tabularx}
\end{table}

\setcounter{taskcounter}{16}
\begin{questions}{\answeringintroduction}
    \question %17
    \begin{subquestions}
        \subquestion \addspace
        \subquestion \addspace
    \end{subquestions}
    \question %18
    \begin{subquestions}
        \subquestion \addspace
        \subquestion \addspace
    \end{subquestions}
    \question %19
    \begin{subquestions}
        \subquestion \addspace
        \subquestion \addspace
    \end{subquestions}
    \question %20
    \begin{subquestions}
        \subquestion \addspace
        \subquestion \addspace
    \end{subquestions}
    \question %21
    \begin{subquestions}
        \subquestion \addspace
        \subquestion \addspace
    \end{subquestions}
    \question %22
    \begin{subquestions}
        \subquestion \addspace
        \subquestion \addspace
        \subquestion \addspace
    \end{subquestions}
    \question %23
    \begin{subquestions}
        \subquestion \addspace
        \subquestion \addspace
        \subquestion \addspace
    \end{subquestions}
    \question %24
    \begin{subquestions}
        \subquestion \addspace
        \subquestion \addspace
        \subquestion \addspace
    \end{subquestions}
\end{questions}
\end{document}\documentclass[10pt]{article}
\usepackage{mathpaper}

\begin{document}
\showsecret
\anspapertitle{第***章~~~章节名}
\informationline \par
\noindent  \ \textbf{\selectingintroduction}
\begin{table}[!htb]
    \centering
    \begin{tabularx}{\textwidth}{|*{11}{>{\centering\arraybackslash}X|}} \hline
        题号 & 1 & 2 & 3 & 4 & 5 & 6 & 7 & 8 & 9 & 10 \\ \hline
        选项 & \quad & \quad & \quad & \quad & \quad & \quad & \quad & \quad & \quad & \quad \\ \hline
    \end{tabularx}
\end{table}
\par \noindent  \ \textbf{\complitingintroduction}
\begin{table}[!htb]
    \centering
    \renewcommand\arraystretch{1.5}
    \begin{tabularx}{\textwidth}{*{3}{>{\centering\arraybackslash}X}}
        11.\complitingline\complitingline\complitingline & 12.\complitingline\complitingline\complitingline & 13.\complitingline\complitingline\complitingline \\
        14.\complitingline\complitingline\complitingline & 15.\complitingline\complitingline\complitingline & 16.\complitingline\complitingline\complitingline  \\
    \end{tabularx}
\end{table}

\setcounter{taskcounter}{16}
\begin{questions}{\answeringintroduction}
    \question %17
    \begin{subquestions}
        \subquestion \addspace
        \subquestion \addspace
    \end{subquestions}
    \question %18
    \begin{subquestions}
        \subquestion \addspace
        \subquestion \addspace
    \end{subquestions}
    \question %19
    \begin{subquestions}
        \subquestion \addspace
        \subquestion \addspace
    \end{subquestions}
    \question %20
    \begin{subquestions}
        \subquestion \addspace
        \subquestion \addspace
    \end{subquestions}
    \question %21
    \begin{subquestions}
        \subquestion \addspace
        \subquestion \addspace
    \end{subquestions}
    \question %22
    \begin{subquestions}
        \subquestion \addspace
        \subquestion \addspace
        \subquestion \addspace
    \end{subquestions}
    \question %23
    \begin{subquestions}
        \subquestion \addspace
        \subquestion \addspace
        \subquestion \addspace
    \end{subquestions}
    \question %24
    \begin{subquestions}
        \subquestion \addspace
        \subquestion \addspace
        \subquestion \addspace
    \end{subquestions}
\end{questions}
\end{document}