\documentclass[10pt]{article}
\usepackage[UTF8]{ctex}
\usepackage[a4paper,left=25mm,right=25mm,top=25mm]{geometry}
\usepackage{xeCJKfntef}
\usepackage{amsmath, amsfonts,amssymb}
\usepackage{markdown}
\markdownSetup{smartEllipses = true}
%开启后可以把markdown中的...转化为省略号
\markdownSetup{hashEnumerators = true}
%有序列表增强,开启后可以采用多种形式书写有序列表
\markdownSetup{inlineFootnotes = true}
%开启后可以添加markdown的超链接和尾注
\markdownSetup{fencedCode = true}
%开启后可以转化markdown的代码块
\markdownSetup{pipeTables = true}
%开启后可以转化markdown的插图和表格
\markdownSetup{hybrid = true}
\markdownSetup{
  renderers = {
    image = {\begin{figure}[htb]
      \centering
      \includegraphics[scale = 1]{#3}%
    \end{figure}},
  }
}
%选择题盒子,1行4选项
\newcommand{\onp} [4] {
    \begin{tabular} {*{4}{@{}p{3.5cm}}}
        A.~#1 & B.~#2 & C.~#3 & D.~#4
    \end{tabular}
}
%选择题盒子,1行2选项
\newcommand{\twp} [4] {
    \begin{tabular} {*{2}{@{}p{7cm}}}
        A.~#1 & B.~#2
    \end{tabular}

    \begin{tabular} {*{2}{@{}p{7cm}}}
        C.~#3 & D.~#4
    \end{tabular}
}
%选择题盒子,1行1选项
\newcommand{\fop} [4] {
    A.~#1

    B.~#2

    C.~#3

    D.~#4
}
\begin{document}
\section*{\centering 试卷标题}
\section*{\normalsize 一、选择题(每小题3分,共30分)}
\begin{markdown}
1. t1

% t1_choice

2. t2

% t2_choice

3. t3

% t3_choice

4. t4

% t4_choice

5. t5

% t5_choice

6. t6

% t6_choice

7. t7

% t7_choice

8. t8

% t8_choice

9. t9

% t9_choice

10. t10

% t10_choice

\end{markdown}
\section*{\normalsize 二、填空题(每小题3分,共18分)}
\begin{markdown}
11. t11

12. t12

13. t13

14. t14

15. t15

16. t16

\end{markdown}
\section*{\normalsize 三、解答题(共8题、72分,每小题应写出文字说明、解答过程或演算步骤)}
\begin{markdown}
17. t17

(1) t17.1

(2) t17.2

18. t18

(1) t18.1

(2) t18.2

19. t19

(1) t19.1

(2) t19.2

20. t20

(1) t20.1

(2) t20.2

21. t21

(1) t21.1

(2) t21.2

(3) t21.3

(4) t21.4

22. t22

(1) t22.1

(2) t22.2

(3) t22.3

23. t23

(1) t23.1

(2) t23.2

(3) t23.3

24. t24

(1) t24.1

(2) t24.2

(3) t24.3

\end{markdown}
\end{document}